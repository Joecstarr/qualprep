\documentclass[14pt]{extarticle}


\usepackage{amsmath, amssymb, amsthm, amsfonts, mathrsfs}
\usepackage{times, flexisym, mdframed, xcolor}
\usepackage{ulem,multicol}
\usepackage{mathtools}
\usepackage{tikz}
\usepackage{hyperref}
\usepackage{graphicx}
\usepackage{fancyhdr}
\usepackage{tikz-cd}%   Margins
% \usepackage[left=1in,right=1in, top=2in, bottom=2in]{geometry}
\usepackage[paperwidth= 20in,paperheight=12in,left=.25in,right=12in, top=.25in, bottom=.25in]{geometry}
%

\usepackage{draculatheme}
\parindent0pt
\begin{document}
$\mathbf{R}$ - I: Solve at your choice ONE of the following problems:

a) Suppose for all any $x \in(0,1)$ and $\varepsilon>0$ there exists $0<r<\varepsilon$, such that $\int_{x-r}^{x+r} f(x) d x \geq 2 r$. Show that $f \geq 1$ a.e for $x \in[0,1]$.

b) Is there a closed, uncountable subset of $\mathbb{R}$ containing no rational numbers? Justify your answer!

c) (True-False) Let $f: \mathbb{R} \rightarrow \mathbb{R}$ be a measurable function and denote by $A=\left\{x \in \mathbb{R}: m\left(f^{-1}(\{x\})\right)>0\right\}$. Then $m(A)=0$. If you believe is true provide a proof otherwise supply a counterexample.
%%%%%%%%%%%%%%%%%%%%%%%%%%%%%%%%%%%%%%%%%%%%%%%%%%%%%%%%%%%%%%%%%%%%%%%%%%%%%%%%%%%%%%%%%%%%%%%%%%%%%%%%%%%%%%%%%%%%%%%%
\newpage
%%%%%%%%%%%%%%%%%%%%%%%%%%%%%%%%%%%%%%%%%%%%%%%%%%%%%%%%%%%%%%%%%%%%%%%%%%%%%%%%%%%%%%%%%%%%%%%%%%%%%%%%%%%%%%%%%%%%%%%%
$\mathbf{R}$ - II: (True-False) If $f$ is integrable on $\mathbb{R}$ then $\lim _{x \rightarrow \infty} f(x)=0$. If you believe it is true provide a proof, otherwise supply a counterexample.
%%%%%%%%%%%%%%%%%%%%%%%%%%%%%%%%%%%%%%%%%%%%%%%%%%%%%%%%%%%%%%%%%%%%%%%%%%%%%%%%%%%%%%%%%%%%%%%%%%%%%%%%%%%%%%%%%%%%%%%%
\newpage
%%%%%%%%%%%%%%%%%%%%%%%%%%%%%%%%%%%%%%%%%%%%%%%%%%%%%%%%%%%%%%%%%%%%%%%%%%%%%%%%%%%%%%%%%%%%%%%%%%%%%%%%%%%%%%%%%%%%%%%%
R - III: Suppose $E$ is a measurable set such that $m(E \cap(a, b)) \geq \frac{b-a}{2}$ for all $a<b$. Show that $E$ is the whole axis except a measure zero set.
%%%%%%%%%%%%%%%%%%%%%%%%%%%%%%%%%%%%%%%%%%%%%%%%%%%%%%%%%%%%%%%%%%%%%%%%%%%%%%%%%%%%%%%%%%%%%%%%%%%%%%%%%%%%%%%%%%%%%%%%
\newpage
%%%%%%%%%%%%%%%%%%%%%%%%%%%%%%%%%%%%%%%%%%%%%%%%%%%%%%%%%%%%%%%%%%%%%%%%%%%%%%%%%%%%%%%%%%%%%%%%%%%%%%%%%%%%%%%%%%%%%%%%
$\mathbf{R}$ - IV: Let $A$ be a measurable subset of $[0,2]$ and define $f: \mathbb{R} \rightarrow \mathbb{R}$ by letting $f(x)=m((-\infty, x] \cap A)$, for every $x \in \mathbb{R}$; here $m$ is the Lebesgue measure on $\mathbb{R}$.

1.) Show that $f$ is absolutely continuous on $\mathbb{R}$, calculate $f^{\prime}$ and $\int_{0}^{3} f^{\prime}(x) d m(x)$, explaining your reasoning.

2.) Show that for every $0<b<m(A)$ there exists $x_{0} \in \mathbb{R}$ such that $b=m\left(\left(-\infty, x_{0}\right] \cap A\right)$.


%%%%%%%%%%%%%%%%%%%%%%%%%%%%%%%%%%%%%%%%%%%%%%%%%%%%%%%%%%%%%%%%%%%%%%%%%%%%%%%%%%%%%%%%%%%%%%%%%%%%%%%%%%%%%%%%%%%%%%%%
\newpage
%%%%%%%%%%%%%%%%%%%%%%%%%%%%%%%%%%%%%%%%%%%%%%%%%%%%%%%%%%%%%%%%%%%%%%%%%%%%%%%%%%%%%%%%%%%%%%%%%%%%%%%%%%%%%%%%%%%%%%%%
$\mathbf{R}$ - V: Let $1 \leq p<\infty$ and suppose that $f, f_{k} \in L^{p}(\mathbb{R})$ are functions satisfying $\lim _{k \rightarrow \infty} f_{k}(x)=f(x)$, for almost every $x \in \mathbb{R}$. Then prove that $\lim _{k \rightarrow \infty}\left\|f_{k}-f\right\|_{L^{p}}=0$ if and only if $\lim _{k \rightarrow \infty}\left\|f_{k}\right\|_{L^{p}}=\|f\|_{L^{p}}$.

%%%%%%%%%%%%%%%%%%%%%%%%%%%%%%%%%%%%%%%%%%%%%%%%%%%%%%%%%%%%%%%%%%%%%%%%%%%%%%%%%%%%%%%%%%%%%%%%%%%%%%%%%%%%%%%%%%%%%%%%
\newpage
%%%%%%%%%%%%%%%%%%%%%%%%%%%%%%%%%%%%%%%%%%%%%%%%%%%%%%%%%%%%%%%%%%%%%%%%%%%%%%%%%%%%%%%%%%%%%%%%%%%%%%%%%%%%%%%%%%%%%%%%
C - I: Solve at your choice ONE of the following problems:

a) If $0<a<1$ then show that
$$
\int_{-\infty}^{\infty} \frac{e^{a x}}{1+e^{x}} d x=\frac{\pi}{\sin (a \pi)}
$$
b) (True-False) Let $f$ be analytic on the open punctured unit disk $D(0,1) \backslash\{0\}$. Can $f^{\prime}$ have a polar singularity of order one at 0 ? If you believe it is true provide a proof, if not supply a counterexample. Also make sure you include all the details in your arguments.

c) Assume that $\left(a_{n}\right)=(1,1,2,3,5,8, \ldots)$ is Fibonacci sequence. Consider the power series $f(z)=$ $\sum_{n} a_{n} z^{n}$. Find the radius of convergence for $f(z)$ and determine a singularity point of the circle of convergence in case it is finite.
%%%%%%%%%%%%%%%%%%%%%%%%%%%%%%%%%%%%%%%%%%%%%%%%%%%%%%%%%%%%%%%%%%%%%%%%%%%%%%%%%%%%%%%%%%%%%%%%%%%%%%%%%%%%%%%%%%%%%%%%
\newpage
%%%%%%%%%%%%%%%%%%%%%%%%%%%%%%%%%%%%%%%%%%%%%%%%%%%%%%%%%%%%%%%%%%%%%%%%%%%%%%%%%%%%%%%%%%%%%%%%%%%%%%%%%%%%%%%%%%%%%%%%
$\mathbf{C}$ - II: Find all entire functions $f$ of finite order such that $f$ has 2020 roots and $f^{\prime}$ has 2022 roots, counted with their multiplicities. State clearly all the theorems you are using.
%%%%%%%%%%%%%%%%%%%%%%%%%%%%%%%%%%%%%%%%%%%%%%%%%%%%%%%%%%%%%%%%%%%%%%%%%%%%%%%%%%%%%%%%%%%%%%%%%%%%%%%%%%%%%%%%%%%%%%%%
\newpage
%%%%%%%%%%%%%%%%%%%%%%%%%%%%%%%%%%%%%%%%%%%%%%%%%%%%%%%%%%%%%%%%%%%%%%%%%%%%%%%%%%%%%%%%%%%%%%%%%%%%%%%%%%%%%%%%%%%%%%%%
$\mathbf{C}$ - III: Assume that $f$ is an entire function such that $|f(z)|=1$ when $|z|=1$. Prove that $f(z)=a z^{n}$ for some integer $n \geq 0$ and some $a \in \mathbb{C}$ with $|a|=1$.
%%%%%%%%%%%%%%%%%%%%%%%%%%%%%%%%%%%%%%%%%%%%%%%%%%%%%%%%%%%%%%%%%%%%%%%%%%%%%%%%%%%%%%%%%%%%%%%%%%%%%%%%%%%%%%%%%%%%%%%%
\newpage
%%%%%%%%%%%%%%%%%%%%%%%%%%%%%%%%%%%%%%%%%%%%%%%%%%%%%%%%%%%%%%%%%%%%%%%%%%%%%%%%%%%%%%%%%%%%%%%%%%%%%%%%%%%%%%%%%%%%%%%%
$\mathbf{C}$ - IV: Let $\mathcal{F}$ be the class of all $f \in H(D(0,1))$ such that $\operatorname{Re} f>0$ and $f(0)=1$. Show $\mathcal{F}$ is a normal family.
%%%%%%%%%%%%%%%%%%%%%%%%%%%%%%%%%%%%%%%%%%%%%%%%%%%%%%%%%%%%%%%%%%%%%%%%%%%%%%%%%%%%%%%%%%%%%%%%%%%%%%%%%%%%%%%%%%%%%%%%
\newpage
%%%%%%%%%%%%%%%%%%%%%%%%%%%%%%%%%%%%%%%%%%%%%%%%%%%%%%%%%%%%%%%%%%%%%%%%%%%%%%%%%%%%%%%%%%%%%%%%%%%%%%%%%%%%%%%%%%%%%%%%
$\mathbf{C}$ - V: Suppose that $f: D(0,1) \rightarrow P$ is a conformal mapping onto a regular pentagonal region $P$, with center at 0 such that $f(0)=0$. Compute $f^{(2020)}(0)$.

(Here we denoted by $f^{(n)}$ the $n$-th derivative of $f$.)

%%%%%%%%%%%%%%%%%%%%%%%%%%%%%%%%%%%%%%%%%%%%%%%%%%%%%%%%%%%%%%%%%%%%%%%%%%%%%%%%%%%%%%%%%%%%%%%%%%%%%%%%%%%%%%%%%%%%%%%%
\newpage
%%%%%%%%%%%%%%%%%%%%%%%%%%%%%%%%%%%%%%%%%%%%%%%%%%%%%%%%%%%%%%%%%%%%%%%%%%%%%%%%%%%%%%%%%%%%%%%%%%%%%%%%%%%%%%%%%%%%%%%%
$\mathbf{R}$ - I:Solve at your choice ONE of the following problems:

a) Suppose for all any $x \in(0,1)$ and $\varepsilon>0$ there exists $0<r<\varepsilon$, such that $\int_{x-r}^{x+r} f(x) d x \geq 2 r$. Show that $f \geq 1$ a.e for $x \in[0,1]$.

b) Is there a closed, uncountable subset of $\mathbb{R}$ containing no rational numbers? Justify your answer!

c) (True-False) Let $f: \mathbb{R} \rightarrow \mathbb{R}$ be a measurable function and denote by $A=\left\{x \in \mathbb{R}: m\left(f^{-1}(\{x\})\right)>0\right\}$. Then $m(A)=0$. If you believe is true provide a proof, otherwise supply a counterexample.

%%%%%%%%%%%%%%%%%%%%%%%%%%%%%%%%%%%%%%%%%%%%%%%%%%%%%%%%%%%%%%%%%%%%%%%%%%%%%%%%%%%%%%%%%%%%%%%%%%%%%%%%%%%%%%%%%%%%%%%%
\newpage
%%%%%%%%%%%%%%%%%%%%%%%%%%%%%%%%%%%%%%%%%%%%%%%%%%%%%%%%%%%%%%%%%%%%%%%%%%%%%%%%%%%%%%%%%%%%%%%%%%%%%%%%%%%%%%%%%%%%%%%%
PROBLEM: R - II:(True-False) If $f$ is integrable on $\mathbb{R}$ then $\lim _{x \rightarrow \infty} f(x)=0$. If you believe it is true provide a proof, otherwise supply a counterexample.



%%%%%%%%%%%%%%%%%%%%%%%%%%%%%%%%%%%%%%%%%%%%%%%%%%%%%%%%%%%%%%%%%%%%%%%%%%%%%%%%%%%%%%%%%%%%%%%%%%%%%%%%%%%%%%%%%%%%%%%%
\newpage
%%%%%%%%%%%%%%%%%%%%%%%%%%%%%%%%%%%%%%%%%%%%%%%%%%%%%%%%%%%%%%%%%%%%%%%%%%%%%%%%%%%%%%%%%%%%%%%%%%%%%%%%%%%%%%%%%%%%%%%%
PROBLEM: R - III: Suppose $E$ is a measurable set such that $m(E \cap(a, b)) \geq \frac{b-a}{2}$ for all $a<b$. Show that $E$ is the whole axis except a measure zero set.




%%%%%%%%%%%%%%%%%%%%%%%%%%%%%%%%%%%%%%%%%%%%%%%%%%%%%%%%%%%%%%%%%%%%%%%%%%%%%%%%%%%%%%%%%%%%%%%%%%%%%%%%%%%%%%%%%%%%%%%%
\newpage
%%%%%%%%%%%%%%%%%%%%%%%%%%%%%%%%%%%%%%%%%%%%%%%%%%%%%%%%%%%%%%%%%%%%%%%%%%%%%%%%%%%%%%%%%%%%%%%%%%%%%%%%%%%%%%%%%%%%%%%%
PROBLEM: $\mathbf{R}$ - IV: Let $A$ be a measurable subset of $[0,2]$ and $\operatorname{define} f: \mathbb{R} \rightarrow \mathbb{R}$ by letting $f(x)=$ $m((-\infty, x] \cap A)$, for every $x \in \mathbb{R}$; here $m$ is the Lebesgue measure on $\mathbb{R}$.

1.) Show that $f$ is absolutely continuous on $\mathbb{R}$, calculate $f^{\prime}$ and $\int_{0}^{3} f^{\prime}(x) d m(x)$, explaining your reasoning.

2.) Show that for every $0<b<m(A)$ there exists $x_{0} \in \mathbb{R}$ such that $b=m\left(\left(-\infty, x_{0}\right] \cap A\right)$.

Make sure you state correctly all the theorems you use in the proof.


%%%%%%%%%%%%%%%%%%%%%%%%%%%%%%%%%%%%%%%%%%%%%%%%%%%%%%%%%%%%%%%%%%%%%%%%%%%%%%%%%%%%%%%%%%%%%%%%%%%%%%%%%%%%%%%%%%%%%%%%
\newpage
%%%%%%%%%%%%%%%%%%%%%%%%%%%%%%%%%%%%%%%%%%%%%%%%%%%%%%%%%%%%%%%%%%%%%%%%%%%%%%%%%%%%%%%%%%%%%%%%%%%%%%%%%%%%%%%%%%%%%%%%
PROBLEM: $\mathbf{R}-\mathbf{V}:$ Let $1 \leq p<\infty$ and suppose that $f, f_{k} \in L^{p}(\mathbb{R})$ are functions satisfying lim $k \rightarrow \infty f_{k}(x)=$ $f(x)$, for almost every $x \in \mathbb{R}$. Then prove that $\lim _{k \rightarrow \infty}\left\|f_{k}-f\right\|_{L^{p}}=0$ if and only if $\lim _{k \rightarrow \infty}\left\|f_{k}\right\|_{L^{p}}=\|f\|_{L^{p}}$. Make sure you state correctly all the theorems you use in the proof.


%%%%%%%%%%%%%%%%%%%%%%%%%%%%%%%%%%%%%%%%%%%%%%%%%%%%%%%%%%%%%%%%%%%%%%%%%%%%%%%%%%%%%%%%%%%%%%%%%%%%%%%%%%%%%%%%%%%%%%%%
\newpage
%%%%%%%%%%%%%%%%%%%%%%%%%%%%%%%%%%%%%%%%%%%%%%%%%%%%%%%%%%%%%%%%%%%%%%%%%%%%%%%%%%%%%%%%%%%%%%%%%%%%%%%%%%%%%%%%%%%%%%%%
PROBLEM: C - I: Solve at your choice ONE of the following problems:

a) If $0<a<1$ then show that
$$
\int_{-\infty}^{\infty} \frac{e^{a x}}{1+e^{x}} d x=\frac{\pi}{\sin (a \pi)}
$$
b) (True-False) Let $f$ be analytic on the open punctured unit disk $D(0,1) \backslash\{0\}$. Can $f^{\prime}$ have a polar singularity of order one at 0? If you believe it is true provide a proof, if not supply a counterexample. Also make sure you include all the details in your arguments.

c) Assume that $\left(a_{n}\right)=(1,1,2,3,5,8, \ldots)$ is Fibonacci sequence. Consider the power series $f(z)=$ $\sum_{n} a_{n} z^{n}$. Find the radius of convergence for $f(z)$ and determine a singularity point of the circle of convergence in case it is finite.



%%%%%%%%%%%%%%%%%%%%%%%%%%%%%%%%%%%%%%%%%%%%%%%%%%%%%%%%%%%%%%%%%%%%%%%%%%%%%%%%%%%%%%%%%%%%%%%%%%%%%%%%%%%%%%%%%%%%%%%%
\newpage
%%%%%%%%%%%%%%%%%%%%%%%%%%%%%%%%%%%%%%%%%%%%%%%%%%%%%%%%%%%%%%%%%%%%%%%%%%%%%%%%%%%%%%%%%%%%%%%%%%%%%%%%%%%%%%%%%%%%%%%%
PROBLEM: C - II. Find all entire functions $f$ of finite order such that $f$ has 2020 roots and $f^{\prime}$ has 2022 roots, counted with their multiplicities. State clearly all the theorems you are using.



%%%%%%%%%%%%%%%%%%%%%%%%%%%%%%%%%%%%%%%%%%%%%%%%%%%%%%%%%%%%%%%%%%%%%%%%%%%%%%%%%%%%%%%%%%%%%%%%%%%%%%%%%%%%%%%%%%%%%%%%
\newpage
%%%%%%%%%%%%%%%%%%%%%%%%%%%%%%%%%%%%%%%%%%%%%%%%%%%%%%%%%%%%%%%%%%%%%%%%%%%%%%%%%%%%%%%%%%%%%%%%%%%%%%%%%%%%%%%%%%%%%%%%
PROBLEM: C - III. Assume that $f$ is an entire function such that $|f(z)|=1$ when $|z|=1$. Prove that $f(z)=a z^{n}$ for some integer $n \geq 0$ and some $a \in \mathbb{C}$ with $|a|=1$.



%%%%%%%%%%%%%%%%%%%%%%%%%%%%%%%%%%%%%%%%%%%%%%%%%%%%%%%%%%%%%%%%%%%%%%%%%%%%%%%%%%%%%%%%%%%%%%%%%%%%%%%%%%%%%%%%%%%%%%%%
\newpage
%%%%%%%%%%%%%%%%%%%%%%%%%%%%%%%%%%%%%%%%%%%%%%%%%%%%%%%%%%%%%%%%%%%%%%%%%%%%%%%%%%%%%%%%%%%%%%%%%%%%%%%%%%%%%%%%%%%%%%%%
PROBLEM: C - IV. Let $\mathcal{F}$ be the class of all $f \in H(D(0,1))$ such that $R e f>0$ and $f(0)=1$. Show $\mathcal{F}$ is a normal family.




%%%%%%%%%%%%%%%%%%%%%%%%%%%%%%%%%%%%%%%%%%%%%%%%%%%%%%%%%%%%%%%%%%%%%%%%%%%%%%%%%%%%%%%%%%%%%%%%%%%%%%%%%%%%%%%%%%%%%%%%
\newpage
%%%%%%%%%%%%%%%%%%%%%%%%%%%%%%%%%%%%%%%%%%%%%%%%%%%%%%%%%%%%%%%%%%%%%%%%%%%%%%%%%%%%%%%%%%%%%%%%%%%%%%%%%%%%%%%%%%%%%%%%
PROBLEM: C - V. Suppose that $f: D(0,1) \rightarrow P$ is a conformal mapping onto a regular pentagonal region $P$, with center at 0 such that $f(0)=0$. Compute $f^{(2020)}(0)$.

(Here we denoted by $f^{(n)}$ the $n$-th derivative of $f$.)




%%%%%%%%%%%%%%%%%%%%%%%%%%%%%%%%%%%%%%%%%%%%%%%%%%%%%%%%%%%%%%%%%%%%%%%%%%%%%%%%%%%%%%%%%%%%%%%%%%%%%%%%%%%%%%%%%%%%%%%%
\newpage
%%%%%%%%%%%%%%%%%%%%%%%%%%%%%%%%%%%%%%%%%%%%%%%%%%%%%%%%%%%%%%%%%%%%%%%%%%%%%%%%%%%%%%%%%%%%%%%%%%%%%%%%%%%%%%%%%%%%%%%%

Let $f$ be a real-valued function defined on the interval $[0,1]$. True-False? If $f$ is not of bounded variation on $[0,1]$, then there is a point $x_{0}$ in $[0,1]$ such that on any open interval $I$ about $x_{0}, f$ fails to be of bounded variation on $I$.
%%%%%%%%%%%%%%%%%%%%%%%%%%%%%%%%%%%%%%%%%%%%%%%%%%%%%%%%%%%%%%%%%%%%%%%%%%%%%%%%%%%%%%%%%%%%%%%%%%%%%%%%%%%%%%%%%%%%%%%%
\newpage
%%%%%%%%%%%%%%%%%%%%%%%%%%%%%%%%%%%%%%%%%%%%%%%%%%%%%%%%%%%%%%%%%%%%%%%%%%%%%%%%%%%%%%%%%%%%%%%%%%%%%%%%%%%%%%%%%%%%%%%%

A (parametrized) curve $C$ in the plane is given by a pair of real-valued functions $f$ and $g$ defined on an interval $[a, b]$. (So as a point set, $C=\{(f(t), g(t)) \mid t \in[a, b]\}$.) The length of $C$ is defined to be

$$
\sup \left\{\sum_{i=1}^{n}\left[\left(f\left(t_{i}\right)-f\left(t_{i-1}\right)\right)^{2}+\left(g\left(t_{i}\right)-g\left(t_{i-1}\right)\right)^{2}\right]^{\frac{1}{2}}\right\}
$$
where the sup is taken over all partitions $a=t_{0}<t_{1}<t_{2}<\cdots<t_{n-1}<t_{n}=b$. True-False? The length of $C$ is finite if and only if $f$ and $g$ are of bounded variation.
%%%%%%%%%%%%%%%%%%%%%%%%%%%%%%%%%%%%%%%%%%%%%%%%%%%%%%%%%%%%%%%%%%%%%%%%%%%%%%%%%%%%%%%%%%%%%%%%%%%%%%%%%%%%%%%%%%%%%%%%
\newpage
%%%%%%%%%%%%%%%%%%%%%%%%%%%%%%%%%%%%%%%%%%%%%%%%%%%%%%%%%%%%%%%%%%%%%%%%%%%%%%%%%%%%%%%%%%%%%%%%%%%%%%%%%%%%%%%%%%%%%%%%

Let $\left\{U_{n}\right\}_{n=1}^{\infty}$ be a sequence of open sets in $[0,1]$. True-False? If the interior of $K:=\cap_{n=1}^{\infty} U_{n}$ is empty, then the Lebesgue measure of $K$ is zero.
%%%%%%%%%%%%%%%%%%%%%%%%%%%%%%%%%%%%%%%%%%%%%%%%%%%%%%%%%%%%%%%%%%%%%%%%%%%%%%%%%%%%%%%%%%%%%%%%%%%%%%%%%%%%%%%%%%%%%%%%
\newpage
%%%%%%%%%%%%%%%%%%%%%%%%%%%%%%%%%%%%%%%%%%%%%%%%%%%%%%%%%%%%%%%%%%%%%%%%%%%%%%%%%%%%%%%%%%%%%%%%%%%%%%%%%%%%%%%%%%%%%%%%

Let $f$ be a non-negative real-valued function defined on the interval $[0,1]$. True-False? $f$ is measurable if and only if there is a (finite or infinite) sequence $\left\{E_{n}\right\}$ of measurable subsets of $[0,1]$ and a sequence of non-negative constants $\left\{c_{n}\right\}$ such that $f(x)=\sum c_{n} 1_{E_{n}}(x)$ for every $x \in[0,1]$.
%%%%%%%%%%%%%%%%%%%%%%%%%%%%%%%%%%%%%%%%%%%%%%%%%%%%%%%%%%%%%%%%%%%%%%%%%%%%%%%%%%%%%%%%%%%%%%%%%%%%%%%%%%%%%%%%%%%%%%%%
\newpage
%%%%%%%%%%%%%%%%%%%%%%%%%%%%%%%%%%%%%%%%%%%%%%%%%%%%%%%%%%%%%%%%%%%%%%%%%%%%%%%%%%%%%%%%%%%%%%%%%%%%%%%%%%%%%%%%%%%%%%%%
Let $\left\{f_{n}\right\}_{n \geq 1}$ be a sequence of non-negative Lebesgue measurable functions defined on $\mathbb{R}$ that converges almost everywhere (with respect to Lebesgue measure $m$ ) to the function $f$. True-False? If $\lim _{n \rightarrow \infty} \int_{\mathbb{R}} f_{n} d m=0$, then $f=0$ a.e. with respect to $m$.
%%%%%%%%%%%%%%%%%%%%%%%%%%%%%%%%%%%%%%%%%%%%%%%%%%%%%%%%%%%%%%%%%%%%%%%%%%%%%%%%%%%%%%%%%%%%%%%%%%%%%%%%%%%%%%%%%%%%%%%%
\newpage
%%%%%%%%%%%%%%%%%%%%%%%%%%%%%%%%%%%%%%%%%%%%%%%%%%%%%%%%%%%%%%%%%%%%%%%%%%%%%%%%%%%%%%%%%%%%%%%%%%%%%%%%%%%%%%%%%%%%%%%%
Let $P_{n}(z):=\sum_{k=0}^{n-1}(k+1) z^{k}, n=1,2, \cdots$ and let $0<r<1$. True-False? There is an $n_{0}$ such that for all $n>n_{0}, P_{n}$ has no zero in the $\operatorname{disc}\{|z|<r\}$.

Suppose $a$ is an isolated singularity of $f$ and suppose the real part of $f, \mathfrak{R}(f(z))$, satisfies the inequality $\mathfrak{R}(f(z)) \leq-m \ln |z-a|$ for some positive integer $m$ and for $z$ in some disc centered at $a$. What kind of singularity is $a$ ? (Is it removable, a pole, or essential?)
%%%%%%%%%%%%%%%%%%%%%%%%%%%%%%%%%%%%%%%%%%%%%%%%%%%%%%%%%%%%%%%%%%%%%%%%%%%%%%%%%%%%%%%%%%%%%%%%%%%%%%%%%%%%%%%%%%%%%%%%
\newpage
%%%%%%%%%%%%%%%%%%%%%%%%%%%%%%%%%%%%%%%%%%%%%%%%%%%%%%%%%%%%%%%%%%%%%%%%%%%%%%%%%%%%%%%%%%%%%%%%%%%%%%%%%%%%%%%%%%%%%%%%

Let $C$ be the circle $x^{2}+y^{2}=2 x$ oriented in the counter clockwise direction. Calculate $\int_{C} \frac{d z}{z^{4}+1}$.
%%%%%%%%%%%%%%%%%%%%%%%%%%%%%%%%%%%%%%%%%%%%%%%%%%%%%%%%%%%%%%%%%%%%%%%%%%%%%%%%%%%%%%%%%%%%%%%%%%%%%%%%%%%%%%%%%%%%%%%%
\newpage
%%%%%%%%%%%%%%%%%%%%%%%%%%%%%%%%%%%%%%%%%%%%%%%%%%%%%%%%%%%%%%%%%%%%%%%%%%%%%%%%%%%%%%%%%%%%%%%%%%%%%%%%%%%%%%%%%%%%%%%%
Let $\Omega$ be a region in the plain, $\mathbb{C}$, and let $\mathcal{F}_{\Omega}=\cup_{n \geq 0}\left\{f|f| \Omega=z^{n}\right\}$. Identify all the regions $\Omega$ such that $\mathcal{F}_{\Omega}$ is a normal family.
%%%%%%%%%%%%%%%%%%%%%%%%%%%%%%%%%%%%%%%%%%%%%%%%%%%%%%%%%%%%%%%%%%%%%%%%%%%%%%%%%%%%%%%%%%%%%%%%%%%%%%%%%%%%%%%%%%%%%%%%
\newpage
%%%%%%%%%%%%%%%%%%%%%%%%%%%%%%%%%%%%%%%%%%%%%%%%%%%%%%%%%%%%%%%%%%%%%%%%%%%%%%%%%%%%%%%%%%%%%%%%%%%%%%%%%%%%%%%%%%%%%%%%

10. Suppose the series $\sum_{k=0}^{\infty} b_{n} z^{n}$ converges in the open unit disc and that $b_{n} \geq 0$ for all $n$. Let

$$
\mathcal{F}=\left\{\sum_{k=0}^{\infty} a_{n} z^{n}|| a_{n} \mid \leq b_{n}\right\}
$$
True-False? $\mathcal{F}$ is a normal family.
%%%%%%%%%%%%%%%%%%%%%%%%%%%%%%%%%%%%%%%%%%%%%%%%%%%%%%%%%%%%%%%%%%%%%%%%%%%%%%%%%%%%%%%%%%%%%%%%%%%%%%%%%%%%%%%%%%%%%%%%
\newpage
%%%%%%%%%%%%%%%%%%%%%%%%%%%%%%%%%%%%%%%%%%%%%%%%%%%%%%%%%%%%%%%%%%%%%%%%%%%%%%%%%%%%%%%%%%%%%%%%%%%%%%%%%%%%%%%%%%%%%%%%

$\mathbf{R}$ - I: Solve at your choice ONE of the following problems:

a) Let $E \subseteq \mathbb{R}$ be a Lebesgue measurable set such that $3 \mu(E \cap(a, b)) \leq b-a$ for all $a<b$. Find $\mu(E)$. Make sure you include all the details in your arguments.

b) Let $f: \mathbb{R} \rightarrow \mathbb{R}$ be an increasing function. Show that $f$ has at most countably many discontinuity points. Conversely, if $A \subset \mathbb{R}$ is a countable subset then there exists a increasing function $f$ whose discontinuity points coincide with $A$. Make sure you include all the details in your arguments.
%%%%%%%%%%%%%%%%%%%%%%%%%%%%%%%%%%%%%%%%%%%%%%%%%%%%%%%%%%%%%%%%%%%%%%%%%%%%%%%%%%%%%%%%%%%%%%%%%%%%%%%%%%%%%%%%%%%%%%%%
\newpage
%%%%%%%%%%%%%%%%%%%%%%%%%%%%%%%%%%%%%%%%%%%%%%%%%%%%%%%%%%%%%%%%%%%%%%%%%%%%%%%%%%%%%%%%%%%%%%%%%%%%%%%%%%%%%%%%%%%%%%%%

$\mathbf{R}$ - II: Let $p \geq 1$. Assume that $\mathrm{f}$ is an absolute continuous function on any compact interval and moreover $f^{\prime} \in L^{p}(\mathbb{R}, \mu)$. Show that
$$
\sum_{n \in \mathbb{Z}}|f(n+1)-f(n)|^{p}<\infty .
$$
%%%%%%%%%%%%%%%%%%%%%%%%%%%%%%%%%%%%%%%%%%%%%%%%%%%%%%%%%%%%%%%%%%%%%%%%%%%%%%%%%%%%%%%%%%%%%%%%%%%%%%%%%%%%%%%%%%%%%%%%
\newpage
%%%%%%%%%%%%%%%%%%%%%%%%%%%%%%%%%%%%%%%%%%%%%%%%%%%%%%%%%%%%%%%%%%%%%%%%%%%%%%%%%%%%%%%%%%%%%%%%%%%%%%%%%%%%%%%%%%%%%%%%

$\mathbf{R}$ - III: Let $(X, d)$ be a compact metric space and let $f: X \rightarrow X$ be an isometry (i.e. $d(f(x), f(y))=d(x, y)$ for all $x, y \in X$ ). Show that $f$ is a homeomorphism (i.e. it is continuous, invertible, and the inverse is continuous as well).
%%%%%%%%%%%%%%%%%%%%%%%%%%%%%%%%%%%%%%%%%%%%%%%%%%%%%%%%%%%%%%%%%%%%%%%%%%%%%%%%%%%%%%%%%%%%%%%%%%%%%%%%%%%%%%%%%%%%%%%%
\newpage
%%%%%%%%%%%%%%%%%%%%%%%%%%%%%%%%%%%%%%%%%%%%%%%%%%%%%%%%%%%%%%%%%%%%%%%%%%%%%%%%%%%%%%%%%%%%%%%%%%%%%%%%%%%%%%%%%%%%%%%%

$\mathbf{R}$ - IV: Let $f_{n} \in L^{3}((0,1))$ nonnegative functions such that $\left\|f_{n}\right\|_{3}=1$ for all $n$ and $f_{n} \rightarrow 0$ almost everywhere as $n \rightarrow \infty$. Show that $\int_{0}^{1} f_{n} d \mu \rightarrow 0$ as $n \rightarrow \infty$.
%%%%%%%%%%%%%%%%%%%%%%%%%%%%%%%%%%%%%%%%%%%%%%%%%%%%%%%%%%%%%%%%%%%%%%%%%%%%%%%%%%%%%%%%%%%%%%%%%%%%%%%%%%%%%%%%%%%%%%%%
\newpage
%%%%%%%%%%%%%%%%%%%%%%%%%%%%%%%%%%%%%%%%%%%%%%%%%%%%%%%%%%%%%%%%%%%%%%%%%%%%%%%%%%%%%%%%%%%%%%%%%%%%%%%%%%%%%%%%%%%%%%%%
$\mathbf{C}-\mathbf{I}$ : Solve at your choice ONE of the following problems:

a) Compute the following integral
$$
\int_{0}^{\infty} \frac{d x}{1+x^{7}}
$$
b) Construct a conformal map from the unit disk onto the infinite horizontal strip $|\operatorname{Im}(z)|<1$. Make sure you include all the details in your arguments.

c) TRUE-FALSE: Let $f$ be analytic on the open punctured unit disk $\mathbb{D} \backslash\{0\}$. Can $f^{\prime}$ have a polar singularity of order one at 0? Make sure you include all the details in your arguments.
%%%%%%%%%%%%%%%%%%%%%%%%%%%%%%%%%%%%%%%%%%%%%%%%%%%%%%%%%%%%%%%%%%%%%%%%%%%%%%%%%%%%%%%%%%%%%%%%%%%%%%%%%%%%%%%%%%%%%%%%
\newpage
%%%%%%%%%%%%%%%%%%%%%%%%%%%%%%%%%%%%%%%%%%%%%%%%%%%%%%%%%%%%%%%%%%%%%%%%%%%%%%%%%%%%%%%%%%%%%%%%%%%%%%%%%%%%%%%%%%%%%%%%

$\mathbf{C}$ - II: Suppose $f: \mathbb{C} \backslash\{0\} \rightarrow \mathbb{R}$ is a nonconstant, real-valued harmonic function on the punctured plane. Prove that the image of $f$ is all of $\mathbb{R}$.
%%%%%%%%%%%%%%%%%%%%%%%%%%%%%%%%%%%%%%%%%%%%%%%%%%%%%%%%%%%%%%%%%%%%%%%%%%%%%%%%%%%%%%%%%%%%%%%%%%%%%%%%%%%%%%%%%%%%%%%%
\newpage
%%%%%%%%%%%%%%%%%%%%%%%%%%%%%%%%%%%%%%%%%%%%%%%%%%%%%%%%%%%%%%%%%%%%%%%%%%%%%%%%%%%%%%%%%%%%%%%%%%%%%%%%%%%%%%%%%%%%%%%%

$\mathbf{C}$ - III: Suppose $f$ is a holomorphic function on $\{z \in \mathbb{C}|| z \mid<1\}$, the open unit disk, with the property that $\operatorname{Re} f(z)>0$ for every point $z$ in the disk. Prove that $\left|f^{\prime}(0)\right| \leq 2 \operatorname{Re} f(0)$.
%%%%%%%%%%%%%%%%%%%%%%%%%%%%%%%%%%%%%%%%%%%%%%%%%%%%%%%%%%%%%%%%%%%%%%%%%%%%%%%%%%%%%%%%%%%%%%%%%%%%%%%%%%%%%%%%%%%%%%%%
\newpage
%%%%%%%%%%%%%%%%%%%%%%%%%%%%%%%%%%%%%%%%%%%%%%%%%%%%%%%%%%%%%%%%%%%%%%%%%%%%%%%%%%%%%%%%%%%%%%%%%%%%%%%%%%%%%%%%%%%%%%%%

$\mathbf{C}$ - IV: Let $f: \mathbb{C} \rightarrow \mathbb{C}$ be an injective holomorphic function. Show there exists $a, b \in \mathbb{C}$ such that $f(z)=a z+b .$
%%%%%%%%%%%%%%%%%%%%%%%%%%%%%%%%%%%%%%%%%%%%%%%%%%%%%%%%%%%%%%%%%%%%%%%%%%%%%%%%%%%%%%%%%%%%%%%%%%%%%%%%%%%%%%%%%%%%%%%%
\newpage
%%%%%%%%%%%%%%%%%%%%%%%%%%%%%%%%%%%%%%%%%%%%%%%%%%%%%%%%%%%%%%%%%%%%%%%%%%%%%%%%%%%%%%%%%%%%%%%%%%%%%%%%%%%%%%%%%%%%%%%%

$\mathbf{R}$ - I: Solve at your choice ONE of the following problems:

a) Let $E$ be a measurable set such that $0<\mu(E)<\infty$. Show that the set $E-E=\{x-y: x, y \in E\}$ contains an nonempty open interval.

b) Show that if $f: \mathbb{R} \rightarrow \mathbb{R}$ is measurable then the set $\left\{x \in \mathbb{R}: \mu\left(f^{-1}(x)\right)>0\right\}$ has measure zero.
%%%%%%%%%%%%%%%%%%%%%%%%%%%%%%%%%%%%%%%%%%%%%%%%%%%%%%%%%%%%%%%%%%%%%%%%%%%%%%%%%%%%%%%%%%%%%%%%%%%%%%%%%%%%%%%%%%%%%%%%
\newpage
%%%%%%%%%%%%%%%%%%%%%%%%%%%%%%%%%%%%%%%%%%%%%%%%%%%%%%%%%%%%%%%%%%%%%%%%%%%%%%%%%%%%%%%%%%%%%%%%%%%%%%%%%%%%%%%%%%%%%%%%

$\mathbf{R}$ - II: Let $f: \mathbb{R} \rightarrow \mathbb{R}$ be Lebesque integrable on the real line. Show that
$$
\lim _{h \rightarrow 0} \int_{\mathbb{R}}|f(x+h)-f(x)| d \mu(x)=0
$$
%%%%%%%%%%%%%%%%%%%%%%%%%%%%%%%%%%%%%%%%%%%%%%%%%%%%%%%%%%%%%%%%%%%%%%%%%%%%%%%%%%%%%%%%%%%%%%%%%%%%%%%%%%%%%%%%%%%%%%%%
\newpage
%%%%%%%%%%%%%%%%%%%%%%%%%%%%%%%%%%%%%%%%%%%%%%%%%%%%%%%%%%%%%%%%%%%%%%%%%%%%%%%%%%%%%%%%%%%%%%%%%%%%%%%%%%%%%%%%%%%%%%%%

$\mathbf{R}$ - III: Let $(X, d)$ be a compact metric space and let $f: X \rightarrow X$ be a function such that $d(f(x), f(y))<$ $d(x, y)$ for all $x, y \in X$ with $x \neq y$. Show that $f$ has a unique fixed point.
%%%%%%%%%%%%%%%%%%%%%%%%%%%%%%%%%%%%%%%%%%%%%%%%%%%%%%%%%%%%%%%%%%%%%%%%%%%%%%%%%%%%%%%%%%%%%%%%%%%%%%%%%%%%%%%%%%%%%%%%
\newpage
%%%%%%%%%%%%%%%%%%%%%%%%%%%%%%%%%%%%%%%%%%%%%%%%%%%%%%%%%%%%%%%%%%%%%%%%%%%%%%%%%%%%%%%%%%%%%%%%%%%%%%%%%%%%%%%%%%%%%%%%

$\mathbf{R}$ - IV: Let $f:[0,1] \rightarrow \mathbb{R}_{+}$be an integrable function and let $F_{n} \subseteq[0,1]$ be a sequence of measurable sets such that $\int_{F_{n}} f d \mu \rightarrow 0$, as $n \rightarrow \infty$. Show that $\mu\left(F_{n}\right) \rightarrow 0$, as $n \rightarrow \infty$.
%%%%%%%%%%%%%%%%%%%%%%%%%%%%%%%%%%%%%%%%%%%%%%%%%%%%%%%%%%%%%%%%%%%%%%%%%%%%%%%%%%%%%%%%%%%%%%%%%%%%%%%%%%%%%%%%%%%%%%%%
\newpage
%%%%%%%%%%%%%%%%%%%%%%%%%%%%%%%%%%%%%%%%%%%%%%%%%%%%%%%%%%%%%%%%%%%%%%%%%%%%%%%%%%%%%%%%%%%%%%%%%%%%%%%%%%%%%%%%%%%%%%%%

$\mathbf{R}-\mathbf{V}$ : Let $F_{k} \subset[0,1], k \in \mathbb{N}$ be measurable sets, and there exists $\delta>0$ such that $m\left(F_{k}\right) \geq \delta$ for all $k$. Assume the sequence $a_{k} \geq 0$ satisfies
$$
\sum_{k=1}^{\infty} a_{k} \chi_{F_{k}}(x)<\infty \text { for a.e. } x \in[0,1]
$$
Show that
$$
\sum_{k=1}^{\infty} a_{k}<\infty .
$$
Make sure you include all the details in your arguments.
%%%%%%%%%%%%%%%%%%%%%%%%%%%%%%%%%%%%%%%%%%%%%%%%%%%%%%%%%%%%%%%%%%%%%%%%%%%%%%%%%%%%%%%%%%%%%%%%%%%%%%%%%%%%%%%%%%%%%%%%
\newpage
%%%%%%%%%%%%%%%%%%%%%%%%%%%%%%%%%%%%%%%%%%%%%%%%%%%%%%%%%%%%%%%%%%%%%%%%%%%%%%%%%%%%%%%%%%%%%%%%%%%%%%%%%%%%%%%%%%%%%%%%

C - I: Define $D=\{z \in \mathbb{C}:|z|<1\}, D_{+}=\left\{z \in \mathbb{C}:|z-i|^{2}<2\right\}, D_{-}=\left\{z \in \mathbb{C}:|z+i|^{2}<2\right\}$, and let $\Omega=D_{+} \cap D_{-}$. Construct a bi-holomophic map from $\Omega$ to $D$.
%%%%%%%%%%%%%%%%%%%%%%%%%%%%%%%%%%%%%%%%%%%%%%%%%%%%%%%%%%%%%%%%%%%%%%%%%%%%%%%%%%%%%%%%%%%%%%%%%%%%%%%%%%%%%%%%%%%%%%%%
\newpage
%%%%%%%%%%%%%%%%%%%%%%%%%%%%%%%%%%%%%%%%%%%%%%%%%%%%%%%%%%%%%%%%%%%%%%%%%%%%%%%%%%%%%%%%%%%%%%%%%%%%%%%%%%%%%%%%%%%%%%%%

$\mathbf{C}$ - II: Let $f \in \mathbb{C}[z]$ be a polynomial of degree $n$. Let $\alpha_{1}, \ldots, \alpha_{n}$ be roots of $f(z)$ and let $\beta_{1}, \ldots, \beta_{n-1}$ be roots of $f^{\prime}(z)$.

\begin{enumerate}
\item If for all $i \in\{1, \ldots, n\}$ we have $\operatorname{Re}\left(\alpha_{i}\right)>0$ then prove that $\operatorname{Re}\left(\beta_{j}\right)>0$ for all $j \in\{1, \ldots, n-1\}$;

\item If for all $i \in\{1, \ldots, n\}$ and $\left|\alpha_{i}\right|<1$ then prove that $\left|\beta_{j}\right|<1$ for all $j \in\{1, \ldots, n-1\}$.

\end{enumerate}
%%%%%%%%%%%%%%%%%%%%%%%%%%%%%%%%%%%%%%%%%%%%%%%%%%%%%%%%%%%%%%%%%%%%%%%%%%%%%%%%%%%%%%%%%%%%%%%%%%%%%%%%%%%%%%%%%%%%%%%%
\newpage
%%%%%%%%%%%%%%%%%%%%%%%%%%%%%%%%%%%%%%%%%%%%%%%%%%%%%%%%%%%%%%%%%%%%%%%%%%%%%%%%%%%%%%%%%%%%%%%%%%%%%%%%%%%%%%%%%%%%%%%%

$\mathbf{C}$ - III: Let $f(z)$ be a non-constant analytic function on $D_{2}=\{z \in \mathbb{C}:|z|<2\}$. If $|f(z)| \equiv 1$ for all $z$ such that $|z|=1$, then prove that $f$ has at least one zero in $D=\{z \in \mathbb{C}:|z|<1\}$.
%%%%%%%%%%%%%%%%%%%%%%%%%%%%%%%%%%%%%%%%%%%%%%%%%%%%%%%%%%%%%%%%%%%%%%%%%%%%%%%%%%%%%%%%%%%%%%%%%%%%%%%%%%%%%%%%%%%%%%%%
\newpage
%%%%%%%%%%%%%%%%%%%%%%%%%%%%%%%%%%%%%%%%%%%%%%%%%%%%%%%%%%%%%%%%%%%%%%%%%%%%%%%%%%%%%%%%%%%%%%%%%%%%%%%%%%%%%%%%%%%%%%%%

$\mathbf{C}$ - IV: Compute the following integral
$$
\int_{-\infty}^{+\infty}\left(\frac{\sin x}{x}\right)^{3} d x
$$
%%%%%%%%%%%%%%%%%%%%%%%%%%%%%%%%%%%%%%%%%%%%%%%%%%%%%%%%%%%%%%%%%%%%%%%%%%%%%%%%%%%%%%%%%%%%%%%%%%%%%%%%%%%%%%%%%%%%%%%%
\newpage
%%%%%%%%%%%%%%%%%%%%%%%%%%%%%%%%%%%%%%%%%%%%%%%%%%%%%%%%%%%%%%%%%%%%%%%%%%%%%%%%%%%%%%%%%%%%%%%%%%%%%%%%%%%%%%%%%%%%%%%%
$\mathbf{C}-\mathbf{V}$ : Show that all the roots of the equation $e^{z}=3 z^{2}$ in $D=\{z \in \mathbb{C}:|z|<1\}$ are real. Ph.D. Qualifying Examination in Analysis
%%%%%%%%%%%%%%%%%%%%%%%%%%%%%%%%%%%%%%%%%%%%%%%%%%%%%%%%%%%%%%%%%%%%%%%%%%%%%%%%%%%%%%%%%%%%%%%%%%%%%%%%%%%%%%%%%%%%%%%%
\newpage
%%%%%%%%%%%%%%%%%%%%%%%%%%%%%%%%%%%%%%%%%%%%%%%%%%%%%%%%%%%%%%%%%%%%%%%%%%%%%%%%%%%%%%%%%%%%%%%%%%%%%%%%%%%%%%%%%%%%%%%%

Let $A$ and $B$ be two subsets of $[0,1]$ whose union is all of $[0,1]$. Show that $m^{*}(A) \geq 1-m^{*}(B)$. (Here, $m^{*}$ denotes Lebesgue outer measure.)
%%%%%%%%%%%%%%%%%%%%%%%%%%%%%%%%%%%%%%%%%%%%%%%%%%%%%%%%%%%%%%%%%%%%%%%%%%%%%%%%%%%%%%%%%%%%%%%%%%%%%%%%%%%%%%%%%%%%%%%%
\newpage
%%%%%%%%%%%%%%%%%%%%%%%%%%%%%%%%%%%%%%%%%%%%%%%%%%%%%%%%%%%%%%%%%%%%%%%%%%%%%%%%%%%%%%%%%%%%%%%%%%%%%%%%%%%%%%%%%%%%%%%%
Define the following function on $C[0,1] \times C[0,1]: d(f, g)=\int_{0}^{1}|f(x)-g(x)| d x$. Show that $d$ is a metric on $C[0,1]$ and determine whether $C[0,1]$ is complete in this metric.
%%%%%%%%%%%%%%%%%%%%%%%%%%%%%%%%%%%%%%%%%%%%%%%%%%%%%%%%%%%%%%%%%%%%%%%%%%%%%%%%%%%%%%%%%%%%%%%%%%%%%%%%%%%%%%%%%%%%%%%%
\newpage
%%%%%%%%%%%%%%%%%%%%%%%%%%%%%%%%%%%%%%%%%%%%%%%%%%%%%%%%%%%%%%%%%%%%%%%%%%%%%%%%%%%%%%%%%%%%%%%%%%%%%%%%%%%%%%%%%%%%%%%%
Let $A$ be a subset of $\mathbb{R}$ with the property that for each $\epsilon>0$ there are (Lebesgue) measurable sets $B$ and $C$ such that
$B \subset A \subset C$

and $m\left(C \cap B^{c}\right)<\epsilon$. Show that $A$ is measurable. (Here, $m$ denotes Lebesgue measure.)

%%%%%%%%%%%%%%%%%%%%%%%%%%%%%%%%%%%%%%%%%%%%%%%%%%%%%%%%%%%%%%%%%%%%%%%%%%%%%%%%%%%%%%%%%%%%%%%%%%%%%%%%%%%%%%%%%%%%%%%%
\newpage
%%%%%%%%%%%%%%%%%%%%%%%%%%%%%%%%%%%%%%%%%%%%%%%%%%%%%%%%%%%%%%%%%%%%%%%%%%%%%%%%%%%%%%%%%%%%%%%%%%%%%%%%%%%%%%%%%%%%%%%%

True-false: Let $f$ be a non-negative continuous function on $\mathbb{R}$ and suppose $\int_{\mathbb{R}} f(x) d x<\infty$, then $\lim _{|x| \rightarrow \infty} f(x)=0 .$
%%%%%%%%%%%%%%%%%%%%%%%%%%%%%%%%%%%%%%%%%%%%%%%%%%%%%%%%%%%%%%%%%%%%%%%%%%%%%%%%%%%%%%%%%%%%%%%%%%%%%%%%%%%%%%%%%%%%%%%%
\newpage
%%%%%%%%%%%%%%%%%%%%%%%%%%%%%%%%%%%%%%%%%%%%%%%%%%%%%%%%%%%%%%%%%%%%%%%%%%%%%%%%%%%%%%%%%%%%%%%%%%%%%%%%%%%%%%%%%%%%%%%%

True-False: If $\left\{f_{n}\right\}_{n \geq 1}$ is a sequence of (Lebesgue) measurable functions such that $0 \leq f_{1} \leq f_{2} \leq f_{3} \leq \cdots$,

if $\sup \int_{\mathbb{R}} f_{n}(x) d x<\infty$ and if $f(x)=\lim f_{n}(x)$ for all $x$, then $\{x \mid f(x)=\infty\}$ has measure zero. Part II
%%%%%%%%%%%%%%%%%%%%%%%%%%%%%%%%%%%%%%%%%%%%%%%%%%%%%%%%%%%%%%%%%%%%%%%%%%%%%%%%%%%%%%%%%%%%%%%%%%%%%%%%%%%%%%%%%%%%%%%%
\newpage
%%%%%%%%%%%%%%%%%%%%%%%%%%%%%%%%%%%%%%%%%%%%%%%%%%%%%%%%%%%%%%%%%%%%%%%%%%%%%%%%%%%%%%%%%%%%%%%%%%%%%%%%%%%%%%%%%%%%%%%%
Calculate the radius of convergence of the power series
\[\sum_{i=0}^\infty z^{n^2}\]
%%%%%%%%%%%%%%%%%%%%%%%%%%%%%%%%%%%%%%%%%%%%%%%%%%%%%%%%%%%%%%%%%%%%%%%%%%%%%%%%%%%%%%%%%%%%%%%%%%%%%%%%%%%%%%%%%%%%%%%%
\newpage
%%%%%%%%%%%%%%%%%%%%%%%%%%%%%%%%%%%%%%%%%%%%%%%%%%%%%%%%%%%%%%%%%%%%%%%%%%%%%%%%%%%%%%%%%%%%%%%%%%%%%%%%%%%%%%%%%%%%%%%%
Suppose $f$ is analytic in the region $0<|z|<1$ and suppose there is a constant $K$ such that $|f(z)| \leq K|z|^{-\frac{1}{2}}$
there. What kind of isolated singularity does $f$ have at zero? (Please prove your answer.)
%%%%%%%%%%%%%%%%%%%%%%%%%%%%%%%%%%%%%%%%%%%%%%%%%%%%%%%%%%%%%%%%%%%%%%%%%%%%%%%%%%%%%%%%%%%%%%%%%%%%%%%%%%%%%%%%%%%%%%%%
\newpage
%%%%%%%%%%%%%%%%%%%%%%%%%%%%%%%%%%%%%%%%%%%%%%%%%%%%%%%%%%%%%%%%%%%%%%%%%%%%%%%%%%%%%%%%%%%%%%%%%%%%%%%%%%%%%%%%%%%%%%%%
For what values of $z$ does the series
$$
\sum_{n=0}^{\infty} \frac{z^{n}}{1-z^{n}}
$$
converge? Is the sum an analytic function of $z$ ?
%%%%%%%%%%%%%%%%%%%%%%%%%%%%%%%%%%%%%%%%%%%%%%%%%%%%%%%%%%%%%%%%%%%%%%%%%%%%%%%%%%%%%%%%%%%%%%%%%%%%%%%%%%%%%%%%%%%%%%%%
\newpage
%%%%%%%%%%%%%%%%%%%%%%%%%%%%%%%%%%%%%%%%%%%%%%%%%%%%%%%%%%%%%%%%%%%%%%%%%%%%%%%%%%%%%%%%%%%%%%%%%%%%%%%%%%%%%%%%%%%%%%%%
True-false: There is a non-constant entire function $f$ such $f(z+1)=f(z)$ and $f(z+i)=f(z)$ for all $z$ in $\mathbb{C}$.
%%%%%%%%%%%%%%%%%%%%%%%%%%%%%%%%%%%%%%%%%%%%%%%%%%%%%%%%%%%%%%%%%%%%%%%%%%%%%%%%%%%%%%%%%%%%%%%%%%%%%%%%%%%%%%%%%%%%%%%%
\newpage
%%%%%%%%%%%%%%%%%%%%%%%%%%%%%%%%%%%%%%%%%%%%%%%%%%%%%%%%%%%%%%%%%%%%%%%%%%%%%%%%%%%%%%%%%%%%%%%%%%%%%%%%%%%%%%%%%%%%%%%%
True-false: Suppose $\left\{f_{n}\right\}_{n=0}^{\infty}$ is a sequence of functions defined and analytic in the open unit disc, $|z|<1$. Suppose also that the values of each $f_{n}$ are contained in the upper half-plane, i.e., suppose that $\operatorname{Im}\left(\mathrm{f}_{\mathrm{n}}(\mathrm{z})\right) \geq 0$ for all $n$ and for all $z,|z|<1$. Then $\left\{f_{n}\right\}_{n=0}^{\infty}$ is a normal family.
%%%%%%%%%%%%%%%%%%%%%%%%%%%%%%%%%%%%%%%%%%%%%%%%%%%%%%%%%%%%%%%%%%%%%%%%%%%%%%%%%%%%%%%%%%%%%%%%%%%%%%%%%%%%%%%%%%%%%%%%
\newpage
%%%%%%%%%%%%%%%%%%%%%%%%%%%%%%%%%%%%%%%%%%%%%%%%%%%%%%%%%%%%%%%%%%%%%%%%%%%%%%%%%%%%%%%%%%%%%%%%%%%%%%%%%%%%%%%%%%%%%%%%

Let $f$ be a bounded measurable function on the interval $[0,1]$ and let $\epsilon>0$ be given. Then there is a step function $\sigma$ on $[0,1]$ such that $|f(x)-\sigma(x)|<\epsilon$ for all $x \in[0,1]$.
%%%%%%%%%%%%%%%%%%%%%%%%%%%%%%%%%%%%%%%%%%%%%%%%%%%%%%%%%%%%%%%%%%%%%%%%%%%%%%%%%%%%%%%%%%%%%%%%%%%%%%%%%%%%%%%%%%%%%%%%
\newpage
%%%%%%%%%%%%%%%%%%%%%%%%%%%%%%%%%%%%%%%%%%%%%%%%%%%%%%%%%%%%%%%%%%%%%%%%%%%%%%%%%%%%%%%%%%%%%%%%%%%%%%%%%%%%%%%%%%%%%%%%

If $\left\{f_{n}\right\}_{n \in \mathbb{N}}$ is a uniformly bounded sequence of nonnegative, Lebesgue integrable functions on $\mathbb{R}$ such that $\left\{f_{n}\right\}_{n \in \mathbb{N}}$ converges to zero pointwise on $\mathbb{R}$, then $\int_{\mathbb{R}} f_{n}(x) d x \rightarrow 0$.
%%%%%%%%%%%%%%%%%%%%%%%%%%%%%%%%%%%%%%%%%%%%%%%%%%%%%%%%%%%%%%%%%%%%%%%%%%%%%%%%%%%%%%%%%%%%%%%%%%%%%%%%%%%%%%%%%%%%%%%%
\newpage
%%%%%%%%%%%%%%%%%%%%%%%%%%%%%%%%%%%%%%%%%%%%%%%%%%%%%%%%%%%%%%%%%%%%%%%%%%%%%%%%%%%%%%%%%%%%%%%%%%%%%%%%%%%%%%%%%%%%%%%%

Suppose $\left\{f_{n}\right\}_{n \in \mathbb{N}}$ is a uniformly bounded sequence of Lebesgue measurable functions defined on [0,1] and for each $n \in \mathbb{N}$ and $x \in[0,1]$, let $F_{n}(x)=\int_{0}^{x} f_{n}(t) d t$. Then the sequence $\left\{F_{n}\right\}_{n \in \mathbb{N}}$ is equicontinuous on $[0,1]$.
%%%%%%%%%%%%%%%%%%%%%%%%%%%%%%%%%%%%%%%%%%%%%%%%%%%%%%%%%%%%%%%%%%%%%%%%%%%%%%%%%%%%%%%%%%%%%%%%%%%%%%%%%%%%%%%%%%%%%%%%
\newpage
%%%%%%%%%%%%%%%%%%%%%%%%%%%%%%%%%%%%%%%%%%%%%%%%%%%%%%%%%%%%%%%%%%%%%%%%%%%%%%%%%%%%%%%%%%%%%%%%%%%%%%%%%%%%%%%%%%%%%%%%

The composition of two absolutely continuous functions on $\mathbb{R}$ is absolutely continuous; i.e., if $f$ and $g$ are two absolutely continuous functions defined on $\mathbb{R}$, then $f \circ g$ is absolutely continuous.
%%%%%%%%%%%%%%%%%%%%%%%%%%%%%%%%%%%%%%%%%%%%%%%%%%%%%%%%%%%%%%%%%%%%%%%%%%%%%%%%%%%%%%%%%%%%%%%%%%%%%%%%%%%%%%%%%%%%%%%%
\newpage
%%%%%%%%%%%%%%%%%%%%%%%%%%%%%%%%%%%%%%%%%%%%%%%%%%%%%%%%%%%%%%%%%%%%%%%%%%%%%%%%%%%%%%%%%%%%%%%%%%%%%%%%%%%%%%%%%%%%%%%%

If $f$ is a continuous, non-decreasing function defined on $[0,1]$ and if $E \subseteq[0,1]$ is a set of Lebesgue measure zero, then $f(E)$ is a set of Lebesgue measure zero.
%%%%%%%%%%%%%%%%%%%%%%%%%%%%%%%%%%%%%%%%%%%%%%%%%%%%%%%%%%%%%%%%%%%%%%%%%%%%%%%%%%%%%%%%%%%%%%%%%%%%%%%%%%%%%%%%%%%%%%%%
\newpage
%%%%%%%%%%%%%%%%%%%%%%%%%%%%%%%%%%%%%%%%%%%%%%%%%%%%%%%%%%%%%%%%%%%%%%%%%%%%%%%%%%%%%%%%%%%%%%%%%%%%%%%%%%%%%%%%%%%%%%%%
The equation $\sin (z)=2$ has no solutions in the complex plane.
%%%%%%%%%%%%%%%%%%%%%%%%%%%%%%%%%%%%%%%%%%%%%%%%%%%%%%%%%%%%%%%%%%%%%%%%%%%%%%%%%%%%%%%%%%%%%%%%%%%%%%%%%%%%%%%%%%%%%%%%
\newpage
%%%%%%%%%%%%%%%%%%%%%%%%%%%%%%%%%%%%%%%%%%%%%%%%%%%%%%%%%%%%%%%%%%%%%%%%%%%%%%%%%%%%%%%%%%%%%%%%%%%%%%%%%%%%%%%%%%%%%%%%

Suppose $f$ is analytic in a region $G$ (in the complex plane) and that for some positive integer $n$, the $n^{t h}$ derivative of $f$ achieves its maximum modulus at a point $z_{0}$ in $G$. Then $f$ is a polynomial of degree at most $n$.
%%%%%%%%%%%%%%%%%%%%%%%%%%%%%%%%%%%%%%%%%%%%%%%%%%%%%%%%%%%%%%%%%%%%%%%%%%%%%%%%%%%%%%%%%%%%%%%%%%%%%%%%%%%%%%%%%%%%%%%%
\newpage
%%%%%%%%%%%%%%%%%%%%%%%%%%%%%%%%%%%%%%%%%%%%%%%%%%%%%%%%%%%%%%%%%%%%%%%%%%%%%%%%%%%%%%%%%%%%%%%%%%%%%%%%%%%%%%%%%%%%%%%%

Let $G$ be a region in the complex plane and let $z_{0}$ be a point in $G$. Suppose that $f$ is a function defined and analytic on $G \backslash\left\{z_{0}\right\}$ and that $f$ maps $G \backslash\left\{z_{0}\right\}$ into the upper half-plane. Then $z_{0}$ is a removable singularity of $f$.
%%%%%%%%%%%%%%%%%%%%%%%%%%%%%%%%%%%%%%%%%%%%%%%%%%%%%%%%%%%%%%%%%%%%%%%%%%%%%%%%%%%%%%%%%%%%%%%%%%%%%%%%%%%%%%%%%%%%%%%%
\newpage
%%%%%%%%%%%%%%%%%%%%%%%%%%%%%%%%%%%%%%%%%%%%%%%%%%%%%%%%%%%%%%%%%%%%%%%%%%%%%%%%%%%%%%%%%%%%%%%%%%%%%%%%%%%%%%%%%%%%%%%%

The function $f(z)=\csc (z)$ has a simple pole at $z=0$ and its residue there is 1 .
%%%%%%%%%%%%%%%%%%%%%%%%%%%%%%%%%%%%%%%%%%%%%%%%%%%%%%%%%%%%%%%%%%%%%%%%%%%%%%%%%%%%%%%%%%%%%%%%%%%%%%%%%%%%%%%%%%%%%%%%
\newpage
%%%%%%%%%%%%%%%%%%%%%%%%%%%%%%%%%%%%%%%%%%%%%%%%%%%%%%%%%%%%%%%%%%%%%%%%%%%%%%%%%%%%%%%%%%%%%%%%%%%%%%%%%%%%%%%%%%%%%%%%

Let $G$ be the open upper half-plane and for $z \in G$ define

$$
\psi_{n}(z):=\exp \left\{\frac{i-(z-n)}{i+(z-n)}\right\},
$$
for $z \in G$ and $n \in \mathbb{N}$. Then $\left\{\psi_{n}\right\}_{n \in \mathbb{N}}$ is a normal family in $H(G)$ with no non-constant limit points.
%%%%%%%%%%%%%%%%%%%%%%%%%%%%%%%%%%%%%%%%%%%%%%%%%%%%%%%%%%%%%%%%%%%%%%%%%%%%%%%%%%%%%%%%%%%%%%%%%%%%%%%%%%%%%%%%%%%%%%%%
\newpage
%%%%%%%%%%%%%%%%%%%%%%%%%%%%%%%%%%%%%%%%%%%%%%%%%%%%%%%%%%%%%%%%%%%%%%%%%%%%%%%%%%%%%%%%%%%%%%%%%%%%%%%%%%%%%%%%%%%%%%%%

Let $f$ be a real-valued continuous function mapping $[0,1]$ to $[0,1]$. True-False? $f$ is absolutely continuous if and only if $f$ maps Lebesgue null sets to Lebesgue null sets.
%%%%%%%%%%%%%%%%%%%%%%%%%%%%%%%%%%%%%%%%%%%%%%%%%%%%%%%%%%%%%%%%%%%%%%%%%%%%%%%%%%%%%%%%%%%%%%%%%%%%%%%%%%%%%%%%%%%%%%%%
\newpage
%%%%%%%%%%%%%%%%%%%%%%%%%%%%%%%%%%%%%%%%%%%%%%%%%%%%%%%%%%%%%%%%%%%%%%%%%%%%%%%%%%%%%%%%%%%%%%%%%%%%%%%%%%%%%%%%%%%%%%%%

Suppose $f$ is a non-negative Lebesgue integrable function defined on $[0,1]$. True-False? There is a Lebesgue measurable set $E \subseteq[0,1]$ such that $f=1_{E}$ if and only if $\int_{0}^{1} f^{n} d \mu=\int_{0}^{1} f d \mu$ for all positive integers $n$.
%%%%%%%%%%%%%%%%%%%%%%%%%%%%%%%%%%%%%%%%%%%%%%%%%%%%%%%%%%%%%%%%%%%%%%%%%%%%%%%%%%%%%%%%%%%%%%%%%%%%%%%%%%%%%%%%%%%%%%%%
\newpage
%%%%%%%%%%%%%%%%%%%%%%%%%%%%%%%%%%%%%%%%%%%%%%%%%%%%%%%%%%%%%%%%%%%%%%%%%%%%%%%%%%%%%%%%%%%%%%%%%%%%%%%%%%%%%%%%%%%%%%%%

For what values of $\alpha>0$ is the function $x \rightarrow x^{\alpha}$ absolutely continuous on every bounded subinterval of $[0, \infty)$ ?
%%%%%%%%%%%%%%%%%%%%%%%%%%%%%%%%%%%%%%%%%%%%%%%%%%%%%%%%%%%%%%%%%%%%%%%%%%%%%%%%%%%%%%%%%%%%%%%%%%%%%%%%%%%%%%%%%%%%%%%%
\newpage
%%%%%%%%%%%%%%%%%%%%%%%%%%%%%%%%%%%%%%%%%%%%%%%%%%%%%%%%%%%%%%%%%%%%%%%%%%%%%%%%%%%%%%%%%%%%%%%%%%%%%%%%%%%%%%%%%%%%%%%%

Let $\sigma$ be the function defined by the formula


$$
\sigma(x):= \begin{cases}0, & x \leq 0 \\ 1, & x>0\end{cases}
$$
and let $\sigma^{*}$ be the outer measure determined by $\sigma$. True-False Every subset of $\mathbb{R}$ is measurable with respect to $\sigma^{*}$.
%%%%%%%%%%%%%%%%%%%%%%%%%%%%%%%%%%%%%%%%%%%%%%%%%%%%%%%%%%%%%%%%%%%%%%%%%%%%%%%%%%%%%%%%%%%%%%%%%%%%%%%%%%%%%%%%%%%%%%%%
\newpage
%%%%%%%%%%%%%%%%%%%%%%%%%%%%%%%%%%%%%%%%%%%%%%%%%%%%%%%%%%%%%%%%%%%%%%%%%%%%%%%%%%%%%%%%%%%%%%%%%%%%%%%%%%%%%%%%%%%%%%%%
True-False The function
$$
f(x):=\sum_{k=1}^{\infty} \sin (k x) / k^{m}
$$
is of bounded variation over every finite interval whenever $m>2$.
%%%%%%%%%%%%%%%%%%%%%%%%%%%%%%%%%%%%%%%%%%%%%%%%%%%%%%%%%%%%%%%%%%%%%%%%%%%%%%%%%%%%%%%%%%%%%%%%%%%%%%%%%%%%%%%%%%%%%%%%
\newpage
%%%%%%%%%%%%%%%%%%%%%%%%%%%%%%%%%%%%%%%%%%%%%%%%%%%%%%%%%%%%%%%%%%%%%%%%%%%%%%%%%%%%%%%%%%%%%%%%%%%%%%%%%%%%%%%%%%%%%%%%
Suppose $f$ is holomorphic in the open unit disc $\mathbb{D}$. Suppose also that for each $z \in \mathbb{D}$ there is an integer $n(z)$ such that the derivative $f^{(n(z))}$ vanishes at $z$. True-False? $f$ must be a polynomial.
%%%%%%%%%%%%%%%%%%%%%%%%%%%%%%%%%%%%%%%%%%%%%%%%%%%%%%%%%%%%%%%%%%%%%%%%%%%%%%%%%%%%%%%%%%%%%%%%%%%%%%%%%%%%%%%%%%%%%%%%
\newpage
%%%%%%%%%%%%%%%%%%%%%%%%%%%%%%%%%%%%%%%%%%%%%%%%%%%%%%%%%%%%%%%%%%%%%%%%%%%%%%%%%%%%%%%%%%%%%%%%%%%%%%%%%%%%%%%%%%%%%%%%

True-False? There is a holomorphic function $f$ on the closed unit disc such that $f\left(\frac{1}{n}\right)=\frac{1}{n+2}, n \geq 1$.
%%%%%%%%%%%%%%%%%%%%%%%%%%%%%%%%%%%%%%%%%%%%%%%%%%%%%%%%%%%%%%%%%%%%%%%%%%%%%%%%%%%%%%%%%%%%%%%%%%%%%%%%%%%%%%%%%%%%%%%%
\newpage
%%%%%%%%%%%%%%%%%%%%%%%%%%%%%%%%%%%%%%%%%%%%%%%%%%%%%%%%%%%%%%%%%%%%%%%%%%%%%%%%%%%%%%%%%%%%%%%%%%%%%%%%%%%%%%%%%%%%%%%%

A function $f$ defined and analytic on a region $G$ is said to have a fixed point $z$ in $G$ if $f(z)=z$. If $f$ is analytic in a region that contains the closed unit disc and if $|f(z)|<1$ for all $z,|z|=1$, how many fixed points does $f$ have in the open unit disc?
%%%%%%%%%%%%%%%%%%%%%%%%%%%%%%%%%%%%%%%%%%%%%%%%%%%%%%%%%%%%%%%%%%%%%%%%%%%%%%%%%%%%%%%%%%%%%%%%%%%%%%%%%%%%%%%%%%%%%%%%
\newpage
%%%%%%%%%%%%%%%%%%%%%%%%%%%%%%%%%%%%%%%%%%%%%%%%%%%%%%%%%%%%%%%%%%%%%%%%%%%%%%%%%%%%%%%%%%%%%%%%%%%%%%%%%%%%%%%%%%%%%%%%

True-False? The function of $r$
$$
\varphi(r):=\int_{|z|=r} \frac{\sin z}{z^{2}+1} d z, \quad r \neq 1
$$
can be extended to a continuous function defined on all of $[0, \infty)$.
%%%%%%%%%%%%%%%%%%%%%%%%%%%%%%%%%%%%%%%%%%%%%%%%%%%%%%%%%%%%%%%%%%%%%%%%%%%%%%%%%%%%%%%%%%%%%%%%%%%%%%%%%%%%%%%%%%%%%%%%
\newpage
%%%%%%%%%%%%%%%%%%%%%%%%%%%%%%%%%%%%%%%%%%%%%%%%%%%%%%%%%%%%%%%%%%%%%%%%%%%%%%%%%%%%%%%%%%%%%%%%%%%%%%%%%%%%%%%%%%%%%%%%
10. Recall that a function $f$ defined on the extended complex plane $\widehat{\mathbf{C}}:=\mathbb{C} \cup\{\infty\}$ is said to be meromorphic on $\widehat{\mathbf{C}}$ in case $f$ is meromorphic on $\mathbb{C}$ and $f\left(\frac{1}{z}\right)$ has a non-essential singularity at $z=0$. Show that a nonconstant function that is meromorphic on $\widehat{\mathbb{C}}$ has the same number of zeros and poles in $\widehat{\mathbf{C}}$.
%%%%%%%%%%%%%%%%%%%%%%%%%%%%%%%%%%%%%%%%%%%%%%%%%%%%%%%%%%%%%%%%%%%%%%%%%%%%%%%%%%%%%%%%%%%%%%%%%%%%%%%%%%%%%%%%%%%%%%%%
\newpage
%%%%%%%%%%%%%%%%%%%%%%%%%%%%%%%%%%%%%%%%%%%%%%%%%%%%%%%%%%%%%%%%%%%%%%%%%%%%%%%%%%%%%%%%%%%%%%%%%%%%%%%%%%%%%%%%%%%%%%%%

Suppose $f_{n}(x), f(x)$ are functions in $L^{1}([0,1])$. Suppose $f_{n}(x) \rightarrow f(x)$ for every $x$, is it true that $\int_{(0,1)} f_{n}(x) d x \rightarrow \int_{(0,1)} f(x) d x ?$ Give a proof or a counterexample.
%%%%%%%%%%%%%%%%%%%%%%%%%%%%%%%%%%%%%%%%%%%%%%%%%%%%%%%%%%%%%%%%%%%%%%%%%%%%%%%%%%%%%%%%%%%%%%%%%%%%%%%%%%%%%%%%%%%%%%%%
\newpage
%%%%%%%%%%%%%%%%%%%%%%%%%%%%%%%%%%%%%%%%%%%%%%%%%%%%%%%%%%%%%%%%%%%%%%%%%%%%%%%%%%%%%%%%%%%%%%%%%%%%%%%%%%%%%%%%%%%%%%%%

Suppose $f \in L^{1}\left(R^{1}\right)$ is it true that $\lim _{x \rightarrow \infty} f(x)=0$ ? Give a proof or a counterexample.
%%%%%%%%%%%%%%%%%%%%%%%%%%%%%%%%%%%%%%%%%%%%%%%%%%%%%%%%%%%%%%%%%%%%%%%%%%%%%%%%%%%%%%%%%%%%%%%%%%%%%%%%%%%%%%%%%%%%%%%%
\newpage
%%%%%%%%%%%%%%%%%%%%%%%%%%%%%%%%%%%%%%%%%%%%%%%%%%%%%%%%%%%%%%%%%%%%%%%%%%%%%%%%%%%%%%%%%%%%%%%%%%%%%%%%%%%%%%%%%%%%%%%%

Suppose that a measurable set $E \subset(0,1)$ is such that $m(E \cap(r, s)) \geq$ $\frac{s-r}{4}$ for all rational $0<r<s<1$. Compute $m(E)$.
%%%%%%%%%%%%%%%%%%%%%%%%%%%%%%%%%%%%%%%%%%%%%%%%%%%%%%%%%%%%%%%%%%%%%%%%%%%%%%%%%%%%%%%%%%%%%%%%%%%%%%%%%%%%%%%%%%%%%%%%
\newpage
%%%%%%%%%%%%%%%%%%%%%%%%%%%%%%%%%%%%%%%%%%%%%%%%%%%%%%%%%%%%%%%%%%%%%%%%%%%%%%%%%%%%%%%%%%%%%%%%%%%%%%%%%%%%%%%%%%%%%%%%

Suppose $f_{k}, f$ are functions in $L^{2}([0,1])$ such that $f_{k}(x) \rightarrow f(x)$, a.e., and that $\left\|f_{k}\right\|_{L^{2}} \rightarrow\|f\|_{L^{2}}$. Is it true that $f_{k} \rightarrow f$ in $L^{2}$ ? Give a proof or a counterexample.
%%%%%%%%%%%%%%%%%%%%%%%%%%%%%%%%%%%%%%%%%%%%%%%%%%%%%%%%%%%%%%%%%%%%%%%%%%%%%%%%%%%%%%%%%%%%%%%%%%%%%%%%%%%%%%%%%%%%%%%%
\newpage
%%%%%%%%%%%%%%%%%%%%%%%%%%%%%%%%%%%%%%%%%%%%%%%%%%%%%%%%%%%%%%%%%%%%%%%%%%%%%%%%%%%%%%%%%%%%%%%%%%%%%%%%%%%%%%%%%%%%%%%%

Suppose $f$ is absolutely continuous and that $f^{\prime} \in L^{1}\left(R^{1}\right)$. Prove that $\lim _{x \rightarrow+\infty} f(x)$ exists. Does the limit have to be zero? Complex Analysis. Solve any four problems in this group of five problems.
%%%%%%%%%%%%%%%%%%%%%%%%%%%%%%%%%%%%%%%%%%%%%%%%%%%%%%%%%%%%%%%%%%%%%%%%%%%%%%%%%%%%%%%%%%%%%%%%%%%%%%%%%%%%%%%%%%%%%%%%
\newpage
%%%%%%%%%%%%%%%%%%%%%%%%%%%%%%%%%%%%%%%%%%%%%%%%%%%%%%%%%%%%%%%%%%%%%%%%%%%%%%%%%%%%%%%%%%%%%%%%%%%%%%%%%%%%%%%%%%%%%%%%

Find all entire functions $f$ for which there is a positive number $C$ such that $|f(z)| \leq C(1+|z|)$ for all $z$.
%%%%%%%%%%%%%%%%%%%%%%%%%%%%%%%%%%%%%%%%%%%%%%%%%%%%%%%%%%%%%%%%%%%%%%%%%%%%%%%%%%%%%%%%%%%%%%%%%%%%%%%%%%%%%%%%%%%%%%%%
\newpage
%%%%%%%%%%%%%%%%%%%%%%%%%%%%%%%%%%%%%%%%%%%%%%%%%%%%%%%%%%%%%%%%%%%%%%%%%%%%%%%%%%%%%%%%%%%%%%%%%%%%%%%%%%%%%%%%%%%%%%%%

Prove the uniform limit of a sequence of holomorphic functions is holomorphic.
%%%%%%%%%%%%%%%%%%%%%%%%%%%%%%%%%%%%%%%%%%%%%%%%%%%%%%%%%%%%%%%%%%%%%%%%%%%%%%%%%%%%%%%%%%%%%%%%%%%%%%%%%%%%%%%%%%%%%%%%
\newpage
%%%%%%%%%%%%%%%%%%%%%%%%%%%%%%%%%%%%%%%%%%%%%%%%%%%%%%%%%%%%%%%%%%%%%%%%%%%%%%%%%%%%%%%%%%%%%%%%%%%%%%%%%%%%%%%%%%%%%%%%

Suppose $f$ is holomorphic on the unit disc and satisfies the inequality $|f(z)| \leq(1-|z|)^{-1}$ for all $z$ in the disc. Prove that $\left|f^{\prime}(z)\right| \leq C(1-|z|)^{-2}$ for some constant $C$.
%%%%%%%%%%%%%%%%%%%%%%%%%%%%%%%%%%%%%%%%%%%%%%%%%%%%%%%%%%%%%%%%%%%%%%%%%%%%%%%%%%%%%%%%%%%%%%%%%%%%%%%%%%%%%%%%%%%%%%%%
\newpage
%%%%%%%%%%%%%%%%%%%%%%%%%%%%%%%%%%%%%%%%%%%%%%%%%%%%%%%%%%%%%%%%%%%%%%%%%%%%%%%%%%%%%%%%%%%%%%%%%%%%%%%%%%%%%%%%%%%%%%%%

Suppose $f$ is entire and that its imaginary part satisfies the inequality $\operatorname{Im}(f) \geq 0$. Show $f$ is a constant.
%%%%%%%%%%%%%%%%%%%%%%%%%%%%%%%%%%%%%%%%%%%%%%%%%%%%%%%%%%%%%%%%%%%%%%%%%%%%%%%%%%%%%%%%%%%%%%%%%%%%%%%%%%%%%%%%%%%%%%%%
\newpage
%%%%%%%%%%%%%%%%%%%%%%%%%%%%%%%%%%%%%%%%%%%%%%%%%%%%%%%%%%%%%%%%%%%%%%%%%%%%%%%%%%%%%%%%%%%%%%%%%%%%%%%%%%%%%%%%%%%%%%%%

Suppose $f$ is analytic in the annular region $1 \leq|z| \leq 2$. Suppose also that $|f| \leq 1$ on the circle $|z|=1$ and that $|f| \leq 4$ on the circle $|z|=2$. Show that $|f(z)| \leq|z|^{2}$ for all $z, 1 \leq|z| \leq 2$.
%%%%%%%%%%%%%%%%%%%%%%%%%%%%%%%%%%%%%%%%%%%%%%%%%%%%%%%%%%%%%%%%%%%%%%%%%%%%%%%%%%%%%%%%%%%%%%%%%%%%%%%%%%%%%%%%%%%%%%%%
\newpage
%%%%%%%%%%%%%%%%%%%%%%%%%%%%%%%%%%%%%%%%%%%%%%%%%%%%%%%%%%%%%%%%%%%%%%%%%%%%%%%%%%%%%%%%%%%%%%%%%%%%%%%%%%%%%%%%%%%%%%%%

$\mathbf{R}$ - I: Solve at your choice ONE of the following problems:

a) Let $E$ be the subset of all elements in $[0,1]$ which do not contain the digits 3 and 9 in their decimal expansion. Is $E$ Lebesgue measurable? If yes find its measure.

b) Show that if $f: \mathbb{R} \rightarrow \mathbb{R}$ is measurable then the set $\left\{x \in \mathbb{R}: \mu\left(f^{-1}(x)\right)>0\right\}$ has measure zero.
%%%%%%%%%%%%%%%%%%%%%%%%%%%%%%%%%%%%%%%%%%%%%%%%%%%%%%%%%%%%%%%%%%%%%%%%%%%%%%%%%%%%%%%%%%%%%%%%%%%%%%%%%%%%%%%%%%%%%%%%
\newpage
%%%%%%%%%%%%%%%%%%%%%%%%%%%%%%%%%%%%%%%%%%%%%%%%%%%%%%%%%%%%%%%%%%%%%%%%%%%%%%%%%%%%%%%%%%%%%%%%%%%%%%%%%%%%%%%%%%%%%%%%

$\mathbf{R}$ - II: Let $f_{n}:[-1,1] \rightarrow \mathbb{R}$ be a sequence of Lebesque measurable functions that converges to $f$ almost everywhere. If $\int_{[-1,1]}\left|f_{n}\right|^{4} d \mu \leq 1$ for every $n$ then show that $\int_{[-1,1]}\left|f_{n}-f\right| d \mu$ converges to 0 .
%%%%%%%%%%%%%%%%%%%%%%%%%%%%%%%%%%%%%%%%%%%%%%%%%%%%%%%%%%%%%%%%%%%%%%%%%%%%%%%%%%%%%%%%%%%%%%%%%%%%%%%%%%%%%%%%%%%%%%%%
\newpage
%%%%%%%%%%%%%%%%%%%%%%%%%%%%%%%%%%%%%%%%%%%%%%%%%%%%%%%%%%%%%%%%%%%%%%%%%%%%%%%%%%%%%%%%%%%%%%%%%%%%%%%%%%%%%%%%%%%%%%%%

$\mathbf{R}$ - III: Let $(X, d)$ be a compact metric space and let $f: X \rightarrow X$ be a continuous function. Show that there exists $A \subseteq X$ a compact subset such that $f(A)=A$.
%%%%%%%%%%%%%%%%%%%%%%%%%%%%%%%%%%%%%%%%%%%%%%%%%%%%%%%%%%%%%%%%%%%%%%%%%%%%%%%%%%%%%%%%%%%%%%%%%%%%%%%%%%%%%%%%%%%%%%%%
\newpage
%%%%%%%%%%%%%%%%%%%%%%%%%%%%%%%%%%%%%%%%%%%%%%%%%%%%%%%%%%%%%%%%%%%%%%%%%%%%%%%%%%%%%%%%%%%%%%%%%%%%%%%%%%%%%%%%%%%%%%%%

$\mathbf{R}$ - IV: Let $F_{k} \subset[0,1], k \in \mathbb{N}$ be measurable sets, and there exists $\delta>0$ such that $m\left(F_{k}\right) \geq \delta$ for all $k$. Assume the sequence $a_{k} \geq 0$ satisfies
$$
\sum_{k=1}^{\infty} a_{k} \chi_{F_{k}}(x)<\infty \text { for a.e. } x \in[0,1] \text {. }
$$
Show that
$$
\sum_{k=1}^{\infty} a_{k}<\infty
$$
Make sure you include all the details in your arguments.
%%%%%%%%%%%%%%%%%%%%%%%%%%%%%%%%%%%%%%%%%%%%%%%%%%%%%%%%%%%%%%%%%%%%%%%%%%%%%%%%%%%%%%%%%%%%%%%%%%%%%%%%%%%%%%%%%%%%%%%%
\newpage
%%%%%%%%%%%%%%%%%%%%%%%%%%%%%%%%%%%%%%%%%%%%%%%%%%%%%%%%%%%%%%%%%%%%%%%%%%%%%%%%%%%%%%%%%%%%%%%%%%%%%%%%%%%%%%%%%%%%%%%%
$\mathbf{C}-\mathbf{I}$ : Solve at your choice ONE of the following problems:

a) Compute the following integral
$$
\int_{-\infty}^{+\infty}\left(\frac{\sin x}{x}\right)^{3} d x
$$
b) Let $\mathcal{P}$ be the open region determined by the pentagon with vertices at $\omega^{k}$ where $k=\overline{0,4}$ and $\omega=\cos (2 \pi / 5)+i \sin (2 \pi / 5)$. Let $f: \overline{\mathcal{P}} \rightarrow \mathbb{C}$ be a continuous function that is analytic on $\mathcal{P}$. Assume that for every $t \in(0,1)$ we have that $\lim _{z \rightarrow \frac{2-t+t \omega}{2}} f(z)=\lim _{z \rightarrow \frac{2-t \omega^{2}+t \omega^{3}}{2}} f(z)=0$. Find $f$.
%%%%%%%%%%%%%%%%%%%%%%%%%%%%%%%%%%%%%%%%%%%%%%%%%%%%%%%%%%%%%%%%%%%%%%%%%%%%%%%%%%%%%%%%%%%%%%%%%%%%%%%%%%%%%%%%%%%%%%%%
\newpage
%%%%%%%%%%%%%%%%%%%%%%%%%%%%%%%%%%%%%%%%%%%%%%%%%%%%%%%%%%%%%%%%%%%%%%%%%%%%%%%%%%%%%%%%%%%%%%%%%%%%%%%%%%%%%%%%%%%%%%%%

$\mathbf{C}$ - II: If we denote by $\mathcal{H}=\{z \in \mathbb{C}:|z-i|>1\}$ then describe all analytic, bijective maps $f: \mathcal{H} \rightarrow \mathcal{H}$.
%%%%%%%%%%%%%%%%%%%%%%%%%%%%%%%%%%%%%%%%%%%%%%%%%%%%%%%%%%%%%%%%%%%%%%%%%%%%%%%%%%%%%%%%%%%%%%%%%%%%%%%%%%%%%%%%%%%%%%%%
\newpage
%%%%%%%%%%%%%%%%%%%%%%%%%%%%%%%%%%%%%%%%%%%%%%%%%%%%%%%%%%%%%%%%%%%%%%%%%%%%%%%%%%%%%%%%%%%%%%%%%%%%%%%%%%%%%%%%%%%%%%%%

$\mathbf{C}$ - III: Let $f$ be a non-constant, analytic function on the unit disk $\mathbb{D}$. If there exists a power series expansion $f(z)=\sum_{n=0}^{\infty} a_{n} z^{n}$ such that $\sum_{n=2}^{\infty} n\left|a_{n}\right| \leq\left|a_{1}\right|$ then show that $f$ is injective.
%%%%%%%%%%%%%%%%%%%%%%%%%%%%%%%%%%%%%%%%%%%%%%%%%%%%%%%%%%%%%%%%%%%%%%%%%%%%%%%%%%%%%%%%%%%%%%%%%%%%%%%%%%%%%%%%%%%%%%%%
\newpage
%%%%%%%%%%%%%%%%%%%%%%%%%%%%%%%%%%%%%%%%%%%%%%%%%%%%%%%%%%%%%%%%%%%%%%%%%%%%%%%%%%%%%%%%%%%%%%%%%%%%%%%%%%%%%%%%%%%%%%%%

$\mathbf{C}$ - IV: Let $f$ be an analytic function on the open unit disk $\mathbb{D}$. Assume that for every $z \in(-1,0]$ the power series expansion around $z$ has a vanishing coefficient. Show that $f$ is a a polynomial function.
%%%%%%%%%%%%%%%%%%%%%%%%%%%%%%%%%%%%%%%%%%%%%%%%%%%%%%%%%%%%%%%%%%%%%%%%%%%%%%%%%%%%%%%%%%%%%%%%%%%%%%%%%%%%%%%%%%%%%%%%
\newpage
%%%%%%%%%%%%%%%%%%%%%%%%%%%%%%%%%%%%%%%%%%%%%%%%%%%%%%%%%%%%%%%%%%%%%%%%%%%%%%%%%%%%%%%%%%%%%%%%%%%%%%%%%%%%%%%%%%%%%%%%

Suppose $f_{n}(x), f(x)$ are measurable functions on $(0,1)$. Suppose $f_{n}(x) \rightarrow$ $f(x)$, it is true that $\int_{(0,1)} f_{n}(x) d x \rightarrow \int_{(0,1)} f(x) d x ?$ Give a proof or a counterexample.
%%%%%%%%%%%%%%%%%%%%%%%%%%%%%%%%%%%%%%%%%%%%%%%%%%%%%%%%%%%%%%%%%%%%%%%%%%%%%%%%%%%%%%%%%%%%%%%%%%%%%%%%%%%%%%%%%%%%%%%%
\newpage
%%%%%%%%%%%%%%%%%%%%%%%%%%%%%%%%%%%%%%%%%%%%%%%%%%%%%%%%%%%%%%%%%%%%%%%%%%%%%%%%%%%%%%%%%%%%%%%%%%%%%%%%%%%%%%%%%%%%%%%%

Suppose $f$ is a function defined $[0,1]$ and suppose that $|f(x)-f(y)| \leq$ $M|x-y|$ for all $x, y \in[0,1]$ where $M$ is a fixed constant. Prove that $f$ is differentiable a.e. and $\left|f^{\prime}(x)\right| \leq M$.
%%%%%%%%%%%%%%%%%%%%%%%%%%%%%%%%%%%%%%%%%%%%%%%%%%%%%%%%%%%%%%%%%%%%%%%%%%%%%%%%%%%%%%%%%%%%%%%%%%%%%%%%%%%%%%%%%%%%%%%%
\newpage
%%%%%%%%%%%%%%%%%%%%%%%%%%%%%%%%%%%%%%%%%%%%%%%%%%%%%%%%%%%%%%%%%%%%%%%%%%%%%%%%%%%%%%%%%%%%%%%%%%%%%%%%%%%%%%%%%%%%%%%%

Show that there is no measurable set such that that $m(E \cap(a, b))=\frac{b-a}{2}$ for all $a<b$, here $m(A)$ is the Lebesgue measure of $A$.
%%%%%%%%%%%%%%%%%%%%%%%%%%%%%%%%%%%%%%%%%%%%%%%%%%%%%%%%%%%%%%%%%%%%%%%%%%%%%%%%%%%%%%%%%%%%%%%%%%%%%%%%%%%%%%%%%%%%%%%%
\newpage
%%%%%%%%%%%%%%%%%%%%%%%%%%%%%%%%%%%%%%%%%%%%%%%%%%%%%%%%%%%%%%%%%%%%%%%%%%%%%%%%%%%%%%%%%%%%%%%%%%%%%%%%%%%%%%%%%%%%%%%%

Suppose $f_{k}, f$ are functions in $L^{1}([0,1])$ such that $f_{k}(x) \rightarrow f(x)$, a.e., and that $\left\|f_{k}\right\|_{L^{1}} \rightarrow\|f\|_{L^{1}}$. Then $f_{k} \rightarrow f$ in $L^{1}$.
%%%%%%%%%%%%%%%%%%%%%%%%%%%%%%%%%%%%%%%%%%%%%%%%%%%%%%%%%%%%%%%%%%%%%%%%%%%%%%%%%%%%%%%%%%%%%%%%%%%%%%%%%%%%%%%%%%%%%%%%
\newpage
%%%%%%%%%%%%%%%%%%%%%%%%%%%%%%%%%%%%%%%%%%%%%%%%%%%%%%%%%%%%%%%%%%%%%%%%%%%%%%%%%%%%%%%%%%%%%%%%%%%%%%%%%%%%%%%%%%%%%%%%

If $f \in L^{1}\left(R^{1}\right)$, show that $\sum_{n=-\infty}^{\infty} f(x+n)$ is convergent e.a to a function which has period $1 .$

%%%%%%%%%%%%%%%%%%%%%%%%%%%%%%%%%%%%%%%%%%%%%%%%%%%%%%%%%%%%%%%%%%%%%%%%%%%%%%%%%%%%%%%%%%%%%%%%%%%%%%%%%%%%%%%%%%%%%%%%
\newpage
%%%%%%%%%%%%%%%%%%%%%%%%%%%%%%%%%%%%%%%%%%%%%%%%%%%%%%%%%%%%%%%%%%%%%%%%%%%%%%%%%%%%%%%%%%%%%%%%%%%%%%%%%%%%%%%%%%%%%%%%

Find all entire functions with the condition that $|f(z)| \leq A\left(1+|z|^{2}\right)$ for some constant $A$.
%%%%%%%%%%%%%%%%%%%%%%%%%%%%%%%%%%%%%%%%%%%%%%%%%%%%%%%%%%%%%%%%%%%%%%%%%%%%%%%%%%%%%%%%%%%%%%%%%%%%%%%%%%%%%%%%%%%%%%%%
\newpage
%%%%%%%%%%%%%%%%%%%%%%%%%%%%%%%%%%%%%%%%%%%%%%%%%%%%%%%%%%%%%%%%%%%%%%%%%%%%%%%%%%%%%%%%%%%%%%%%%%%%%%%%%%%%%%%%%%%%%%%%

Compute $\int_{\partial D(0,1)}(1+\bar{z})^{5} d z$.
%%%%%%%%%%%%%%%%%%%%%%%%%%%%%%%%%%%%%%%%%%%%%%%%%%%%%%%%%%%%%%%%%%%%%%%%%%%%%%%%%%%%%%%%%%%%%%%%%%%%%%%%%%%%%%%%%%%%%%%%
\newpage
%%%%%%%%%%%%%%%%%%%%%%%%%%%%%%%%%%%%%%%%%%%%%%%%%%%%%%%%%%%%%%%%%%%%%%%%%%%%%%%%%%%%%%%%%%%%%%%%%%%%%%%%%%%%%%%%%%%%%%%%

Suppose $f$ is holomorphic in the punctured disk $D(0,1) \backslash\{0\}$. Suppose also that $|f| \leq \frac{1}{|z|^{0.5}}$. Prove $f$ is differentiable at 0 .
%%%%%%%%%%%%%%%%%%%%%%%%%%%%%%%%%%%%%%%%%%%%%%%%%%%%%%%%%%%%%%%%%%%%%%%%%%%%%%%%%%%%%%%%%%%%%%%%%%%%%%%%%%%%%%%%%%%%%%%%
\newpage
%%%%%%%%%%%%%%%%%%%%%%%%%%%%%%%%%%%%%%%%%%%%%%%%%%%%%%%%%%%%%%%%%%%%%%%%%%%%%%%%%%%%%%%%%%%%%%%%%%%%%%%%%%%%%%%%%%%%%%%%

Suppose $f$ is entire so that $\operatorname{Re}(f) \geq 0$. Show $f$ is a constant.
%%%%%%%%%%%%%%%%%%%%%%%%%%%%%%%%%%%%%%%%%%%%%%%%%%%%%%%%%%%%%%%%%%%%%%%%%%%%%%%%%%%%%%%%%%%%%%%%%%%%%%%%%%%%%%%%%%%%%%%%
\newpage
%%%%%%%%%%%%%%%%%%%%%%%%%%%%%%%%%%%%%%%%%%%%%%%%%%%%%%%%%%%%%%%%%%%%%%%%%%%%%%%%%%%%%%%%%%%%%%%%%%%%%%%%%%%%%%%%%%%%%%%%

Suppose $f$ is holomorphic in the unit disk. Prove that there exists $z_{n}$ in the disk with $\left|z_{n}\right| \rightarrow 1$ that $f\left(z_{n}\right)$ is bounded.
%%%%%%%%%%%%%%%%%%%%%%%%%%%%%%%%%%%%%%%%%%%%%%%%%%%%%%%%%%%%%%%%%%%%%%%%%%%%%%%%%%%%%%%%%%%%%%%%%%%%%%%%%%%%%%%%%%%%%%%%
\newpage
%%%%%%%%%%%%%%%%%%%%%%%%%%%%%%%%%%%%%%%%%%%%%%%%%%%%%%%%%%%%%%%%%%%%%%%%%%%%%%%%%%%%%%%%%%%%%%%%%%%%%%%%%%%%%%%%%%%%%%%%

True-false. Lebesgue measure is continuous in the sense that the Lebesgue measure of the closure of a set coincides with the Lebesgue measure of the set.
%%%%%%%%%%%%%%%%%%%%%%%%%%%%%%%%%%%%%%%%%%%%%%%%%%%%%%%%%%%%%%%%%%%%%%%%%%%%%%%%%%%%%%%%%%%%%%%%%%%%%%%%%%%%%%%%%%%%%%%%
\newpage
%%%%%%%%%%%%%%%%%%%%%%%%%%%%%%%%%%%%%%%%%%%%%%%%%%%%%%%%%%%%%%%%%%%%%%%%%%%%%%%%%%%%%%%%%%%%%%%%%%%%%%%%%%%%%%%%%%%%%%%%

True-false. Let $I_{1}$ and $I_{2}$ be two disjoint open intervals, and for $i=1,2$, let $A_{i}$ be an arbitrary subset of $I_{i}$. Then $m^{*}\left(A_{1} \cup A_{2}\right)=m^{*}\left(A_{1}\right)+m^{*}\left(A_{2}\right)$, where $m^{*}$ denotes Lebesgue outer measure.
%%%%%%%%%%%%%%%%%%%%%%%%%%%%%%%%%%%%%%%%%%%%%%%%%%%%%%%%%%%%%%%%%%%%%%%%%%%%%%%%%%%%%%%%%%%%%%%%%%%%%%%%%%%%%%%%%%%%%%%%
\newpage
%%%%%%%%%%%%%%%%%%%%%%%%%%%%%%%%%%%%%%%%%%%%%%%%%%%%%%%%%%%%%%%%%%%%%%%%%%%%%%%%%%%%%%%%%%%%%%%%%%%%%%%%%%%%%%%%%%%%%%%%

True-false. Let $\left\{f_{n}\right\}_{n \geq 0}$ be a sequence of non-negative integrable functions defined on $\mathbb{R}$ such that


$$
0 \leq f_{1}(x) \leq f_{2}(x) \leq f_{3}(x) \leq \ldots
$$
and such that the sequence of numbers $\left\{\int_{\mathbb{R}} f_{n}(x) d x\right\}$ is bounded. Let $f(x)=\lim _{n \rightarrow \infty} f_{n}(x)$. Then $f(x)<\infty$ for almost all $x .$
%%%%%%%%%%%%%%%%%%%%%%%%%%%%%%%%%%%%%%%%%%%%%%%%%%%%%%%%%%%%%%%%%%%%%%%%%%%%%%%%%%%%%%%%%%%%%%%%%%%%%%%%%%%%%%%%%%%%%%%%
\newpage
%%%%%%%%%%%%%%%%%%%%%%%%%%%%%%%%%%%%%%%%%%%%%%%%%%%%%%%%%%%%%%%%%%%%%%%%%%%%%%%%%%%%%%%%%%%%%%%%%%%%%%%%%%%%%%%%%%%%%%%%
True-false. Let $\sigma$ be the function on $\mathbb{R}$ that is zero to the left of $1,1 / 2$ for $1 \leq x<2,3 / 4$ for $2 \leq x<3$, $7 / 8$ for $3 \leq x<4$, etc. (Thus, $\sigma$ jumps by $2^{-n}$ at $n$ for $n=1,2, \ldots$, is constant between any two consecutive integers and is continuous from the right.) If $f(x)=x$ on $\mathbb{R}$, then $f$ is integrable with respect to the Lebesgue-Stieltjes measure determined by $\sigma$.
%%%%%%%%%%%%%%%%%%%%%%%%%%%%%%%%%%%%%%%%%%%%%%%%%%%%%%%%%%%%%%%%%%%%%%%%%%%%%%%%%%%%%%%%%%%%%%%%%%%%%%%%%%%%%%%%%%%%%%%%
\newpage
%%%%%%%%%%%%%%%%%%%%%%%%%%%%%%%%%%%%%%%%%%%%%%%%%%%%%%%%%%%%%%%%%%%%%%%%%%%%%%%%%%%%%%%%%%%%%%%%%%%%%%%%%%%%%%%%%%%%%%%%

Let $\left\{f_{n}\right\}$ be a uniformly bounded sequence of measurable functions defined on the interval $[0,1]$ and let

$$
F_{n}(x)=\int_{0}^{x} f_{n}(t) d t \quad 0 \leq x \leq 1
$$
Show that there is a subsequence $\left\{F_{n_{k}}\right\}$ that converges uniformly on $[0,1]$.
%%%%%%%%%%%%%%%%%%%%%%%%%%%%%%%%%%%%%%%%%%%%%%%%%%%%%%%%%%%%%%%%%%%%%%%%%%%%%%%%%%%%%%%%%%%%%%%%%%%%%%%%%%%%%%%%%%%%%%%%
\newpage
%%%%%%%%%%%%%%%%%%%%%%%%%%%%%%%%%%%%%%%%%%%%%%%%%%%%%%%%%%%%%%%%%%%%%%%%%%%%%%%%%%%%%%%%%%%%%%%%%%%%%%%%%%%%%%%%%%%%%%%%

True-false. Suppose $f$ is analytic in the region $0<|z|<1$ and suppose that for each $r, 0<r<1$, the integral $\int_{C_{r}} f(z) d z=0$, where $C_{r}$ is the circle $|z|=r$. Then $f$ is analytic on the open unit disc.
%%%%%%%%%%%%%%%%%%%%%%%%%%%%%%%%%%%%%%%%%%%%%%%%%%%%%%%%%%%%%%%%%%%%%%%%%%%%%%%%%%%%%%%%%%%%%%%%%%%%%%%%%%%%%%%%%%%%%%%%
\newpage
%%%%%%%%%%%%%%%%%%%%%%%%%%%%%%%%%%%%%%%%%%%%%%%%%%%%%%%%%%%%%%%%%%%%%%%%%%%%%%%%%%%%%%%%%%%%%%%%%%%%%%%%%%%%%%%%%%%%%%%%

Suppose $f$ is analytic in the annular region $1-\epsilon<|z|<2+\epsilon$ for some positive $\epsilon$. Suppose also that $|f| \leq 1$ on the circle $|z|=1$ and that $|f| \leq 4$ on the circle $|z|=2$. Show that $|f(z)| \leq|z|^{2}$ for all $z$, $1<|z|<2$
%%%%%%%%%%%%%%%%%%%%%%%%%%%%%%%%%%%%%%%%%%%%%%%%%%%%%%%%%%%%%%%%%%%%%%%%%%%%%%%%%%%%%%%%%%%%%%%%%%%%%%%%%%%%%%%%%%%%%%%%
\newpage
%%%%%%%%%%%%%%%%%%%%%%%%%%%%%%%%%%%%%%%%%%%%%%%%%%%%%%%%%%%%%%%%%%%%%%%%%%%%%%%%%%%%%%%%%%%%%%%%%%%%%%%%%%%%%%%%%%%%%%%%

True-false. Let $\mathfrak{G}$ be a domain and let $z_{0}$ be a point in $\mathfrak{G}$. Suppose $f$ is analytic in $\mathfrak{G} /\left\{z_{0}\right\}$ and that $f$ takes values in the upper half-plane. Then $z_{0}$ is a removable singularity of $f$.
%%%%%%%%%%%%%%%%%%%%%%%%%%%%%%%%%%%%%%%%%%%%%%%%%%%%%%%%%%%%%%%%%%%%%%%%%%%%%%%%%%%%%%%%%%%%%%%%%%%%%%%%%%%%%%%%%%%%%%%%
\newpage
%%%%%%%%%%%%%%%%%%%%%%%%%%%%%%%%%%%%%%%%%%%%%%%%%%%%%%%%%%%%%%%%%%%%%%%%%%%%%%%%%%%%%%%%%%%%%%%%%%%%%%%%%%%%%%%%%%%%%%%%

Find the Laurent series representation of the function

$$
f(z)=\frac{1}{z^{2}(1-z)}
$$
that is valid in the region $1<|z|<\infty$.
%%%%%%%%%%%%%%%%%%%%%%%%%%%%%%%%%%%%%%%%%%%%%%%%%%%%%%%%%%%%%%%%%%%%%%%%%%%%%%%%%%%%%%%%%%%%%%%%%%%%%%%%%%%%%%%%%%%%%%%%
\newpage
%%%%%%%%%%%%%%%%%%%%%%%%%%%%%%%%%%%%%%%%%%%%%%%%%%%%%%%%%%%%%%%%%%%%%%%%%%%%%%%%%%%%%%%%%%%%%%%%%%%%%%%%%%%%%%%%%%%%%%%%
Let $f$ be analytic in the open unit disc $\mathbb{D}$ and let the Taylor series expansion for $f$ be
$$
f(z)=\sum_{n=0}^{\infty} a_{n} z^{n}
$$
Suppose\\
(a) $f(\mathbb{D}) \subseteq \mathbb{D}$\\
(b) $a_{0}=0$\\
(c) $\left|a_{1}\right|=1$

Calculate $\sup \left\{\left|a_{n}\right| \mid n \geq 2\right\}$.
%%%%%%%%%%%%%%%%%%%%%%%%%%%%%%%%%%%%%%%%%%%%%%%%%%%%%%%%%%%%%%%%%%%%%%%%%%%%%%%%%%%%%%%%%%%%%%%%%%%%%%%%%%%%%%%%%%%%%%%%
\newpage
%%%%%%%%%%%%%%%%%%%%%%%%%%%%%%%%%%%%%%%%%%%%%%%%%%%%%%%%%%%%%%%%%%%%%%%%%%%%%%%%%%%%%%%%%%%%%%%%%%%%%%%%%%%%%%%%%%%%%%%%


\end{document}